\documentclass[12pt, a4paper]{book}
\usepackage[ascii]{inputenc}
\usepackage[left=2cm,right=2cm,top=2cm,bottom=4cm]{geometry}
\usepackage[protrusion=true,expansion=true]{microtype}

\usepackage{amsmath}
\usepackage{amsfonts}
\usepackage{amssymb}
\usepackage{tikz, pgfplots}
\usetikzlibrary{intersections}
\usepackage{kpfonts}
\usepackage{dsfont}
\pgfplotsset{compat=1.13}
\usepackage{emptypage}

\DeclareMathOperator{\N}{\mathbb{N}}
\DeclareMathOperator{\Q}{\mathbb{Q}}
\DeclareMathOperator{\Z}{\mathbb{Z}}
\DeclareMathOperator{\R}{\mathbb{R}}
\DeclareMathOperator{\C}{\mathbb{C}}
\DeclareMathOperator{\F}{\mathbb{F}}

\usepackage{graphicx}
\usepackage{enumitem}
\setenumerate{}

%-----------------------------------------------------------------------------------------------------------------
% Some fancy macros // May eventually move these into separate files or something and merge when building template
\renewcommand{\d}[1]{\ensuremath{\operatorname{d}\!{#1}}} % dx macro for integrals
\newcommand{\hess}[1]{\ensuremath{\operatorname{H}\!{#1}}} % Hessian
\newcommand{\diff}[1]{\ensuremath{\operatorname{D}\!{#1}}} % Jacobian
\newcommand{\inner}[2]{\left\langle #1, #2 \right\rangle} % inner product
\newcommand{\norm}[1]{\left\lVert#1\right\rVert} % norm
\newcommand{\cpl}[1]{\overline{#1}} % complement
\renewcommand{\v}[1]{\mathbf{#1}} % vector
\newenvironment{amatrix}[1]{% augumented matrix - make sure to have # columns less than required amount
  \left(\begin{array}{@{}*{#1}{c}|c@{}}
}{%
  \end{array}\right)
}
%-----------------------------------------------------------------------------------------------------------------
% Define theorem environments, along with a custom proof environment
\usepackage[thref, thmmarks,amsmath]{ntheorem}
\newcommand{\itref}[1]{\textit{\thref{#1}}}

\newtheorem{theorem}{Thm.}[section]
\newtheorem{lemma}[theorem]{Lemma}
\newtheorem{definition}[theorem]{Def'n.}
\newtheorem{corollary}[theorem]{Cor.}
\newtheorem{proposition}[theorem]{Prop.}

\theorembodyfont{\upshape}
\newtheorem{remark}[theorem]{Rmk.}
\newtheorem{exercise}[theorem]{Exc.}
\newtheorem{example}[theorem]{Ex.}
\theoremseparator{}
\theoremindent0.0cm
\theoremstyle{nonumberplain}
\theoremheaderfont{\scshape}
\theoremsymbol{$\square$}
\newtheorem{proof}{Proof}

%-----------------------------------------------------------------------------------------------------------------
% Define Document Variables
\newcommand{\assignmentname}{Course Notes}
\newcommand{\classname}{Conjecture and Proof}
\newcommand{\semester}{BSM Fall 2018}

% Define a title page for the document
%----------------------------------------------------------------------------------------------------------------------
% Define headings for each page
\usepackage{fancyheadings}
\pagestyle{fancy}
\lhead{Alex Rutar\\arutar@uwaterloo.ca}
\rhead{\classname: \assignmentname\\\semester}
\cfoot{\thepage}
\setlength{\headheight}{50pt}
%----------------------------------------------------------------------------------------------------------------------
\begin{document}
\pagenumbering{roman}
\begin{titlepage}
    \centering
    \vspace{5cm}
    {\huge\textbf{\assignmentname}\par} % Assignment Name
    \vspace{2cm}
    {\Large\textbf{\classname}\par} % Class
    \vspace{3cm}
    {\Large\textit{Alex Rutar}\par}

    \vfill

% Bottom of the page
    {\large \semester \par} % Due Date
\end{titlepage}
%----------------------------------------------------------------------------------------------------------------------
% \newpage\null\thispagestyle{empty}\textit{This page is left intentionally blank.}\newpage
\pagenumbering{roman}
\tableofcontents
\pagenumbering{arabic}
\chapter{An Introduction}
\section{Sidon Sets}
We define a Sidon set $S\subseteq N$ as a subset such that pairwise sums are unique.
Write $1\leq a_1<a_1<\cdots<a_k\leq n$ with $a_i+a_j\neq a_l+a_r$ (possibly $i=j$, $l=r$).
what is the maximum value of $k$?
For example, the powers of two provide a lower bound of $\max k\geq\lfloor\log_2 n\rfloor+1$ by binary representations and uniqueness of multiplication by 2.

We can also bound above: $2\leq a_i+a_j\leq 2n$ and the number of sums is $\binom{k}{2}+k$.
We must have
\[\binom{k}{2}+k\leq 2n-1\]
which can be rearranged to (losing a small amount of precision)
\[k<2\sqrt{n}\]

We can get a better upper bound: note that if we have equal sums, we also have equal differences: $a_i+a_j=a_l+a_r$ implies $a_i-a_l=a_r-a_j$.
We now have $\binom{k}{2}$ differences and $n-1$ places, and by the same argument as above we get
\[k<\sqrt{2n}+1\]
This trick works because subtraction is not commutative!

Let's now try to get a better lower bound.
Always pick the smallest number available that does not violate the rule.
We can take
\[1,2,4,8,\ldots\]
Assume that we already picked $a_1<a_2<a_3<\cdots<a_l$.
Then can we take $a_{l+1}$: $x$ is bad if $x+a_i=a_j+a_k$, $x+x=a_j+a_k$, $x=a_j+a_k-a_i$ so there are at most $l^3$ bad numbers.
The second is impossible otherwise we would have $x<\max\{a_j,a_k\}$.
Thus there are at most $l^3$ bad numbers, including $a_i=a_i+a_j-a_k$.
Thus if $l^3<n$, we can certainly pick an $a_{l+1}$.
We therefore have
\[\sqrt[3]{n}\leq\max k <\sqrt{2n}+1\]
\section{Irrational Numbers}
\subsection{A few proofs of irrationality}
\begin{proof}
    We provide five different proofs that $\sqrt{5}$ is irrational:
    \begin{enumerate}
        \item By contradiction, suppose $\sqrt{5}=\frac{a}{b}$ with $(a,b)=1$ and $b>0$.
            Then $5b^2=a^2$, so $5|a^2$.
            But since $5$ is prime (or generally, a product of distinct primes), $5|a$ and write $a=5c$ so that $5b^2=(5c)^2=25c^2$.
            But then $b^2=5c^2$ so $5|b$, a contradiction.
        \item As above, get $5b^2=a^2$.
            Using unique factorization in $\Z$, 
            note that $n$ is a square iff $n=p_1^{k_1}\cdots p_l^{k_l}$ and $2|k_i$ for all $i$ (proof is constructive).
            But then $b^2,a^2$ both have an even exponent in the $5$ position, so that $5b^2$ has an odd exponent, a contradiction.

            More generally, if there exists an odd exponent in the standard form of $m$, then $\sqrt{m}$ is irritional.
        \item Suppose $\sqrt{5}=\frac{a}{b}$.
            We must have $\lim_{n\to\infty}(\sqrt{5}-2)^n\to 0$.
            If we multiply $(c+d\sqrt{5})(h+j\sqrt{5})$, we have another number of the same form.
            Then $(\sqrt{5}-2)^n=A_n-B_n\sqrt{5}=A_n+B_n\frac{a}{b}=\frac{C_n}{b}\geq\frac{1}{b}$ with $C_n\neq 0$, contradicting the limit.
        \item In geometry, we say $a$ and $b$ are commesurable (have a common measure) if there exists $c$ so that $kc=a$ and $lc=b$ where $k,l\in\Z$.
            Then $a/b$ is rational if and only if $a,b$ have a common measure.
            To see the forward direction, we have $\frac{a}{b}=\frac{m}{n}$ so that $\frac{a}{m}=\frac{b}{n}$ and a common measure is $\frac{a}{m}$.
            Conversely, if $kc=a$ and $lc=b$ then $\frac{a}{b}=\frac{k}{l}$.

            Thus we will show that $\sqrt{5}$ and $1$ have no common measure.
            Suppose $c$ is a common measure of $1$ and $\sqrt{5}$.
            Consider a rectangle with sides $1,2$ and diagonal of length $\sqrt{5}$.
            Let $AB=1$, $BC=2$ and choose $E$ so that $EC=BC$.
            Drop a perpindicular from $E$ onto $AB$.
            Then $AEF\sim ABC$ since they share two angles.
            But then $FE=2AE$.
            Then $c$ is also a common measure of $FE$.
            Similarly, $FB=FE$ since $FBC\cong FEC$.
            Then $c$ is also a common measure of $FB$ and thus of $AF$.

            Repeat this construction, so we must have $c$ arbitrarily small because the ratios of the hypotenuses are a constant ratio less than $1$.
            Thus we have our contradiction.
        \item $\sqrt{5}$ is a root of the polynomial $x^2-5$.
            We have the rational root test, which states that possible rational roots must 
            Write $f=a_0+a_1x+\cdots+a_nx^n$.
            Consider a root of the form $r/2$, so $f(r/s)=0$. Then
            \[0=a_0s^n+a_1rs^{n-1}+a_2r^2s^{n-2}+\cdots+a_nr^n\]
            so $s|a_nr^n$ so $s|a_n$ (since $(s,r)=1$).
            Similarly, $r|a_0$.

            If $\sqrt{5}=1/b$, then $a|-5$ and $b|1$ so $a/b=\pm 1,\pm 5$.
            Check, and none of these work, so there are no rational roots.
    \end{enumerate}
\end{proof}
\begin{proposition}
    $e$ is irrational.
\end{proposition}
\begin{proof}
    Assume $e=\frac{a}{b}$, $b>0$, $(a,b)=1$  and write
    \[\frac{a}{b}=e=1+\frac{1}{1!}+\frac{1}{2!}+\frac{1}{3!}+\cdots\]
    and muliply by $b!$ to get
    \[\text{integer}=\text{integer}+\frac{1}{b+1}+\frac{1}{(b+1)(b+2)}+\cdots \]
    but the infinite sum is positive less than $\frac{1}{2}+\frac{1}{4}+\cdots=1$, a contradiction.
\end{proof}
\begin{proposition}
    $\sin 1^\circ$ is irrational.
\end{proposition}
\begin{proof}
    We show that if $\sin 1^\circ$ is rational, then $\sin 45^\circ$ is rational.
    Write $z=\cos 1^\circ+i\sin 1^\circ$ so that $z^{45}=(\cos 1^\circ+i\sin 1^\circ)^{45}=\cos 45^\circ+i\sin 45^\circ$.
    Expand the binomial coefficient to get
    \begin{align*}
        \sum\limits_{n=0}^{45}\binom{45}{n}(\cos 1^\circ)^n(i\sin 1^\circ)^{45-n} &= \text{real}+\sum\limits_{\substack{n=0\\2|n}}^{45}\binom{45}{n}(\cos 1^\circ)^n(i\sin 1^\circ)^{45-n}\\
                                                                                  &= \text{real}+i\sum\limits_{\substack{n=0\\2|n}}^{45}(\pm1)\binom{45}{n}(\cos 1^\circ)^n(\sin 1^\circ)^{45-n}
    \end{align*}
    but since $(\cos 1^\circ)^2=1-(\sin 1^\circ)^2$ is rational, the entire imaginary part is rational.
    Thus equating with $\sin 45^\circ$ means that $\sin 45^\circ=\sqrt{2}/2$ is rational, our contradiction.
\end{proof}
\subsection{Algebraic Numbers}
It is interesting to consider numbers which are roots of polynomials with rational (equiv. integer) coefficients of degree at least 1.
The rational numbers $\frac{a}{b}$ are roots of the degree one polynomials $x-\frac{a}{b}$.
\begin{definition}
    We say that $\alpha\in\C$ is algebraic if there exists $p\in \Z[x]$, $p\neq 0$, so that $p(\alpha)=0$.
    If $\alpha$ is not algebraic, then it is transcendental.
\end{definition}
\begin{definition}
    We say that $f$ is the minimal polynomial of $\alpha$ if $f(\alpha)=0$ and $f$ has minimal degree.
\end{definition}
\begin{definition}
    With this in mind, we define the \textbf{degree} of an algebraic number $\deg\alpha=\deg m_\alpha$.
\end{definition}    
We have the following properties of the minimal polynomial:
\begin{theorem}
    The following hold:
\begin{enumerate}[label=(\alph*)]
        \item The minimal polynomial is unique up to a constant factor.
        \item $g(\alpha)=0\Leftrightarrow m_\alpha|g$
        \item $g=m_\alpha\Leftrightarrow g(\alpha)=0$ and $g$ is irreducible over $\Q$, i.e. $g$ cannot be factored into polynomials of smaller degree with rational coefficients.
        \item The algebraic numbers form a subfield of the complex numbers.
    \end{enumerate}
\end{theorem}
\begin{proof}
    We first show $(b)$.
    If $m_\alpha|g$, then $g(\alpha)=m_\alpha(\alpha)f(\alpha)=0$.
    For the reverse direction, write $g=m_\alpha\cdot q+r$ where $\deg r<\deg m_\alpha$.
    Then $0=g(\alpha)=m_\alpha(\alpha)\cdot q+r(\alpha)$ so $r(\alpha)=0$.
    But since $m_\alpha$ is the minimal polynomial, we must have $r=0$ and $m_\alpha|g$.

    Now we see $(a)$ from $(b)$.
    Suppose $p,q$ are both minimal polynomials.
    Then $p|q$ so $q=hp$, where $\deg q=\deg p$.
    Thus $\deg h=0$ is a constant polynomial.

    Now we see $(c)$.
    We certainly have $g(\alpha)=0$.
    Now suppose for contradiction that $g$ is reducible, and write $g=f\cdot h$.
    But then $f(\alpha)h(\alpha)=0$, so w.l.o.g. $f(\alpha)=0$ with $\deg f<\deg g$, so $g$ is not minimal.
    Conversely, $m_\alpha|g$ so $m_\alpha=cg$.
\end{proof}
\begin{example}
    Show that $\deg\sqrt[3]{2}=3$.
    By (c), it suffices to show that $x^3-2$ is irreducible, which follows by the rational root test.

    Now consider $f=x^4-2$, and suppose $f=g\cdot h$.
    $g$ and $h$ cannot be degree 1 by the rational root theorem, but we could have $\deg g=\deg h=2$.
    To prove this, we use the Eisenstein criterion with $p=2$.
\ multiplication by $i$
\end{example}
\begin{theorem}[Gelfond-Schneider]
    Suppose $0,1\neq\alpha$ is algebraic, and $\beta$ is algebraic, and not rational.
    Then $\alpha^\beta$ is transcendental.
\end{theorem}
\begin{corollary}
    $\beta=\log_{10}3$ is transcendental.
\end{corollary}
\begin{proof}
    Write $10^\beta=3$.
    Suppose $\beta$ is algebraic.
    $\beta$ is certainly irrational, but then $10^\beta$ is transcendental, a contradiction.
\end{proof}
\section{Constructing the irrationals}
Let $\alpha\in\R$, $\frac{r}{s}\in\Q$.
We want to find
\[\left\lvert\alpha-\frac{r}{s}\right\rvert<\frac{1}{f(s)}\]
We always assume $(r,s)=1$, $s>0$.
\subsection{Linear Diophantine Equations}
First suppose $\alpha=a/b$.
Then
\[\left\lvert\frac{a}{b}-\frac{r}{s}\right\rvert=\frac{|sa-rb|}{bs}\geq\frac{1}{bs}\]
where equality holds when $sa-rb=\pm1$.
This is an example of a linear diophantine equation: we wish to solve $Ax+By=C$ for integers $A,B,C,x,y$.
\begin{proposition}
    $Ax+By=C$ is solvable if and only if $(A,B)|C$.
    If it is solvable, there are infinitely many solutions.
\end{proposition}
\begin{proof}
    If it is solvable, we have $x_0,y_0$ so $Ax_0+By_0=C$.
    Then $(A,B)$ divides $A$ and $B$ so it must divide a linear combination of $A$ and $B$, so it must also divide $C$.

    The reverse direction is a consequence of the Euclidean algorithm.

    Now suppose we have a solution $Ax_0+By_0=C$, then $A(x_0+tB)+B(y_0-tA)=C$ is also a solution.
\end{proof}
\begin{theorem}
    If $\alpha$ is irrational, then there exists infinitely many $\frac{r}{s}$ so that
    \[\left\lvert\alpha-\frac{r}{s}\right\rvert<\frac{1}{s^2}\]
\end{theorem}
\begin{lemma}
    Let $\alpha\in\R$, $u>0$ an integer.
    Then there exists $r/s$ so that $|\alpha-r/s|<1/(su)$ for $s\leq u$.
\end{lemma}
\begin{proof}
    Define $\{\beta\}=\beta-\lfloor\beta\rfloor$.
    Clearly $0\leq\{\beta\}<1$.
    Thus $0\leq0,\{\alpha\},\{2\alpha\},\ldots,\{n\alpha\}<1$.
    Partition $[0,1)$ into intervals $[a/n,(a+1)/n)$ for $a\leq n-1$.
    Then by the pidgeonhole principle, there exists $i,j$ so that $|\{j\alpha\}-\{i\alpha\}|<1/n$.
    Thus
    \begin{equation*}
        \left\lvert (j-i)\alpha-(\lfloor j\alpha\rfloor-\lfloor i\alpha\rfloor)\right\rvert<\frac{1}{n}
    \end{equation*}
    and take $s=j-i$ and $r=\lfloor j\alpha\rfloor-\lfloor i\alpha\rfloor$ so that
    \begin{equation*}
        \left\lvert\alpha-\frac{r}{s}\right\rvert<\frac{1}{ns}
    \end{equation*}
    showing the lemma.
\end{proof}
\begin{proof}
    Now, let's prove the theorem.
    First, choose $n_1$ and get
    \[\left\lvert|\alpha-\frac{r_1}{s_1}\right\rvert<\frac{1}{u_1s_1}<\frac{1}{s_1^2}\]
    Now repeat with some new choice of $n_2$, to get some $r_2/s_2$.
    Fix $d=|\alpha-r_1/s_1|$.
    In order to guarantee $|\alpha-r_2/s_2|<d$, choose $n_2$ so that $\frac{1}{n_2}<d$, and since $d>0$ ($\alpha$ is irrational), this is always possible.
    Then
    \[\left\lvert\alpha-\frac{r_2}{s_2}\right\rvert <\frac{1}{s_2n_2}<\frac{1}{n_2}<d\]

    As a side note, if we find $r,s$ not relatively prime, write $m=(r,s)$ and $r=mr'$, $s=ms'$.
    Then
    \[\left\lvert\alpha-\frac{r'}{s'}\right\rvert<\frac{1}{m^2s'^2}<\frac{1}{s'^2}\]
\end{proof}
Now, suppose we fix a given $s$.
Then at most how many $r$ can occur?
Note that $\frac{k}{s}<\alpha<\frac{k+1}{s}$.
Then we cannot have $r=k$ and $r=k+1$: if so,
\begin{align*}
    \left\lvert\alpha-\frac{k}{s}\right\rvert&<\frac{1}{s^2}\\
    \left\lvert\alpha-\frac{k+1}{s}\right\rvert&<\frac{1}{s^2}
\end{align*}
so we must have $\frac{2}{s^2}<s$.
Thus if $s>1$, then $r$ is unique, and if $s=1$, then there are two values of $r$.
Thus
\[\lim_{k\to\infty}\left\lvert\alpha-\frac{r_k}{s_k}\right\rvert=0\]
for
\[\left\lvert\alpha-\frac{r_k}{s_k}\right\rvert<\frac{1}{s_k^2}\]
\begin{corollary}
    If $\alpha$ is irrational, and consider the sequence $\{0\},\{\alpha\},\{2\alpha\},\ldots,\{n\alpha\},\ldots$.
    This is dense in $[0,1]$.
\end{corollary}
\begin{proof}
    From the lemma, we have $|s\alpha-r|<1/s$, so as $s\to\infty$, $|s\alpha-r|\to 0$.
    Thus $s\alpha$ is close to an integer, so $\{s\alpha\}$ is close to $0$ or $1$.
    Now $\{2s\alpha\}=2s\alpha+2\lfloor s\alpha\rfloor+2\{s_\alpha\}=2\lfloor s_\alpha\rfloor+\{2s\alpha\}$ as long as $2\{s\alpha\}<1$.
    But then the collection $\{ns\alpha\}$ is within $\epsilon$ of any point on $[0,1]$.
\end{proof}
\begin{theorem}
    If $\deg\alpha=n$, then there exists $c=c(\alpha)>0$ so that, for any $r/s\in\Q$,
    \[\left\lvert\alpha-\frac{r}{s}\right\rvert>\frac{c}{s^n}\]
\end{theorem}
\begin{proof}
    Let $m_\alpha=a_0+a_1x+\cdots+a_nx^n$ with $a_n\neq 0$, $a_i\in\Z$.
    Then over $\C$,
    \begin{align*}
        m_\alpha &= a_0+a_1x+\cdots+a_nx^n\\
                 &= a_n(x-\alpha)(x-\alpha_2)\cdots(x-\alpha_n)
    \end{align*}
    Thus
    \begin{align*}
        \left\lvert m_\alpha\left(\frac{r}{s}\right)\right\rvert &= \left\lvert a_0+a_1\frac{r}{s}+\cdots+a_n\left(\frac{r}{s}\right)^n\right\rvert\\
                                                     &= \left\lvert a_n\left(\frac{r}{s}-\alpha\right)\left(\frac{r}{s}-\alpha_2\right)\cdots\left(\frac{r}{s}-\alpha_n\right)\right\rvert
    \end{align*}
    Now suppose for all $c>0$, there exists $r/s$ so that $|\alpha-r/s|<c/s^n$.
    Then for each $1/2^k$, we have $r_k/s_k$ so that
    \[\left\lvert\alpha-\frac{r_k}{s_k}\right\rvert<\frac{1}{2^ks_k^n}\Leftrightarrow\left\lvert s_k^n\left(\alpha-\frac{r_k}{s_k}\right)\right\rvert<\frac{1}{2^k}\]
    But also recall that
    \[\left\lvert a_0+\frac{r}{s}+\cdots+a_n\left(\frac{r}{s}\right)^n\right\rvert = \frac{\text{integer}}{s^n}\]
    so
    \begin{align*}
        \frac{1}{s_k^n}\leq \left\lvert m_\alpha\left(\frac{r_k}{s_k}\right)\right\rvert &= \left\lvert a_n\left(\frac{r_k}{s_k}-\alpha\right)\left(\frac{r_k}{s_k}-\alpha_2\right)\cdots\left(\frac{r_k}{s_k}-\alpha_n\right)\right\rvert\\
                                                                                         &= \left\lvert \left(\frac{r_k}{s_k}-\alpha\right)g\left(\frac{r_k}{s_k}\right)\right\rvert
    \end{align*}
    and
    \[1\leq \left\lverts_k^n\left(\frac{r_k}{s_k}-\alpha\right)\right\rvert\left\lvert g\left(\frac{r_k}{s_k}\right)\right\rvert\]
    a contradiction, since the right hand side goes to 0.
\end{proof}
To construct a transcendental number, consider
\[\alpha=\sum\limits_{j=1}^\infty\frac{1}{v_j}\]
and define
\[\frac{r_k}{s_k}=\sum\limits_{j=1}^k\frac{1}{v_j}\]
Assume $v_1,\ldots,v_k$ satisfy $|\alpha-r_j/s_j|<1/s_j^r$.
Choose $v_{k+1}$ so that
\[\left\lvert\alpha-\frac{r_k}{s_k}\right\rvert<\frac{1}{s_k^k}\]
or equivalently, $v_{j+1}>2v_j$.
Choose as well $2s^k<v_{k+1}$.
Then
\[\left\lvert\alpha-\frac{r_k}{s_k}\right\rvert=\frac{1}{v_{k+1}}+\frac{1}{v_{k+2}}+\cdots<\frac{2}{v_{k+1}}<\frac{1}{s_k^k}\]
\begin{theorem}
    For any $\delta>0$, the set of $\alpha$ that satisfy ``$\exists$ infinitely many $\frac{r}{s}$ so that $|\alpha-r/s|<1/s^{2+\delta}$'' has measure 0.
    If $\alpha$ is algebraic, there is only finitely many $r/s$ satisfying the property.
\end{theorem}
\section{Diophantine Equations}
Let's consider solutions to $x^2-dy^2=1$.
If $d$ is a perfect square, then the only solutions that exist are the trivial solutions.
However, we do have the following theorem:
\begin{theorem}
    If $d>1$ and $d\neq k^2$, then there are infinitely many solutions to $x^2-dy^2$.
\end{theorem}
\begin{proof}
    First, note the connection to Pell's equation.
    Suppose $x^2-dy^2=1$ has infinitely many solutions $x=r_n$, $y=s_n$.
    Then
    \[r_n^2-ds_n^2=1\Longrightarrow \left(\frac{r_n}{s_n}-\sqrt{d}\right)\left(\frac{r_n}{s_n}+\sqrt{d}\right)=\frac{1}{s_n^2}\]
    so that
    \[\left\lvert\sqrt{d}-\frac{r_n}{s_n}\right\rvert<\frac{1}{s_n^2}\]
    Now suppose we have $m\neq 0$ and infinitely many solution $u_n/v_n$ to $u_n^2-dv_n^2=m$ and $u_1^2-dv_1^2=m$.
    Divide these equations to get
    \[\frac{(u_n+\dqrt{d}v_n)(u_n-\sqrt{d}v_n)}{u_1+\sqrt{d}v_1)(u_1-\sqrt{d}v_1)}\]
    and
    \[(t_1+\sqrt{d}w_1)(t_1-\sqrt{d}w_1)=1\]
    for rational $t_1,w_1$.
    Then one can show that for some $u,v$, they must yield integer solutions.
\end{proof}
\begin{corollary}
    With $d>1$ and $d\neq k^2$, then there are either no solutions or infinitely many solutions to $x^2-dy^2=n$.
\end{corollary}
Interestingly, there are only finitely many solutions to $x^3-dy^3=1$, or similarly for any higher power sums.
This follows since $\sqrt[3]{d}$ cannot be approximated well enough.
\begin{theorem}
    Suppose $1\leq a_1<a_2<\cdots<a_k$ such that all possible subset sums are distinct.
    Prove that
    \[\sum\limits_{j=1}^k\frac{1}{a_j}<2\]
\end{theorem}
\begin{proof}
    Consider $(1+x^{a_1})(1+x^{a_2})(1+x^{a_k})$; and expanding, every coefficient will be 1.
    Thus
    \begin{align*}
        (1+x^{a_1})(1+x^{a_2})(1+x^{a_k}) &< 1+x+x^2+\cdots+x^k+\cdots\\
                                          &= \frac{1}{1-x}
    \end{align*}
    for $0<x<1$.
    In general,
    \begin{align*}
        \int_0^1\frac{\log(1+x^a)}{x}\d{x}&=\int_0^1\frac{\log(1+y)}{ax^{a-1}x}\\
                                          &= \frac{1}{a}\int_0^1\frac{\log(1+y)}{y}\d{y}
    \end{align*}
    Thus taking logarithms of both sides, we have
    \begin{align*}
        \sum\limits_{j=1}^k\frac{\log(1+x^{a_j})}{x}\d{x} &< -\int_0^1\frac{\log(1-x)}{x}\d{x}\\
        \left(\sum\limits_{j=1}^k\frac{1}{a_j}\right)\int_0^1\frac{\log(1+y)}{y}\d{y}&<-\int_0^1\frac{\log(1-x)}{x}\d{x}\\
        \left(\sum\limits_{j=1}^k\frac{1}{a_j}\right)A&<B
    \end{align*}
    We want to show $B=2A$.
    Indeed,
    \begin{align*}
        A-B &= \int_0^1\frac{\log(1+x)+\log(1-x)}{x}\d{x}\\
            &= \int_0^1\frac{\log(1-x^2)}{x}\d{x}\\
            &= \int_0^1\frac{\log(1-y)}{2y}\d{y}\\
            &=-\frac{B}{2}
    \end{align*}
    and rearranging, we get $B=2A$.
\end{proof}
Here's another proof:
\begin{proof}
    We first note that $a_1+\cdots+a_j\geq 2^j-1$, since this must be at least the number of sums formed by $a_1,\ldots,a_j$.
    There are several cases.
    If equality holds for all $j$, then $a_j=2^j-1$.
    Now let $a_1+\cdots+a_r>2^r-1$ where the inequality holds strictly.
    If we substitute $a_r'=a_r-1$, this works of all later equalities hold strictly.
    If not, there is some $s$ so we have equality; in this case, replace $a_s'=a_s+1$.
    As well, we certainly have
    \[\frac{1}{a_r-1}+\frac{1}{a_s+1}>\frac{1}{a_r}+\frac{1}{a_s}\]
    which can be verified by taking reciprocals.
    Thus under any of the substitutions, we have
    \[\sum\frac{1}{a_j}<\sum\frac{1}{a_j'}\]
    and under repeated substitution, we obtain equality everywhere so this is entirely less than 2.
\end{proof}
\chapter{Cardinality}
%-%-%------------------------------------------------------------------------------------------------------------------
\section{Principles}
Cardinality is a way of thinking about the size of a set.
\begin{definition}
    Two sets $A$ and $B$ have the same \textbf{cardinality} if there is a bijection between the sets. If this is the case, we
    say that $|A|=|B|$. If there exists an injection, then we say $|A|\leq|B|$.
\end{definition}
In particular, cardinality is an equivalence relation.
\begin{enumerate}
    \item Reflexive: $|A|\sim|A|$ by the identity map.
    \item Symmetric: If $f:A\to B$ is a bijection, then $f^{-1}:B\to A$ is also a bijection.
    \item Transitive: If $f:A\to B$ and $g:B\to C$ are bijections, then $\phi=g\circ f:A\to C$ is a bijection.
\end{enumerate}
If $A\subseteq B$, then $|A|\leq|B|$ since the embedding maps are injective (the identity function restricted to $A$).
For example, we have $|\N|\leq|\Z|\leq|\Q|\leq|\R|$. We also have $|\N|=|\Z|$ from the bijection given, say, by $f:\Z\to\N$
defined by
\[f(n)=
\begin{cases}
    2n & n>0\\
    -2n+1 & n\leq 0
\end{cases}
\]
which is also listed below.
\begin{center}
    \begin{tabular}{cccccccc}
        $\Z$&0&1&-1&2&-2&3&$\cdots$\\
        $\N$&1&2&3 &4&5 &6&$\cdots$
    \end{tabular}
\end{center}
\begin{definition}
    A set $A$ is countable if $A$ is finite or countably infinite. $A$ is countably infinite if $|A|=|\N|$.
\end{definition}
Countable sets can be ``listed''. If $A$ is finite, we can write $A=\{a_1,\dots,a_n\}$ for some $n\in\N$. If $A$ is countably
infinite, then there exists a bijection $U:\N\to A$ that lets us write
\[A=\{U(i):i\in\N\}\]
and write $a_i=U(i)$. On the other hand, if $A=\{a_i:i\in\N\}$, we have our bijection $f:A\to\N$ given by $a_i\mapsto i$.

\section{Cardinality Examples}
\begin{enumerate}
    \item $\N\times\N=\{(a,b):a,b\in\N\}$. We have $|\N\times\N|=|\N|$.
    \item $|\Q|=|\N|$.
\end{enumerate}
\begin{proposition} The following hold:
    \begin{enumerate}[label=(\arabic*)]
        \item Every infinite subset of $\N$ is countably infinite.
        \item If $A$ is infinite and $|A|\leq|\N|$, then $|\N|=|A|$.
    \end{enumerate}
\end{proposition}
\begin{proof} Prove (1), (2) separately:
    \begin{enumerate}[label=(\arabic*)]
        \item We use the well-ordering property of $\N$: every non-empty subset of $\N$ has a least element. Let $B$
            be an infinite subset of $\N$, so it is non-empty. Thus $B$ has some least element $b_1$. But then, $B\setminus\{b_1\}$
            is also non-empty, so we can repeat this process to create an increasing sequence
            \[b_1<b_2<b_3<\cdots<\]
            I claim that every element of $B$ is in this set. Let $b\in B$ and consider $\{n\in B:n\leq b\}$. This set is
            finite with, say, $k$ elements, so $b=b_k$. We then get our bijection by the standard map $b_i\mapsto i$.
        \item Assume $j:A\to\N$ is an injection. Let $B=j(A)\subseteq\N$. Notice $j:A\to B$ is a bijection, so $|A|=|B|$
            and $B$ is infinite. By (1), $B$ is countably infinite, so $|B|=|\N|$, and the result follows by transitivity.
    \end{enumerate}
\end{proof}

\section{Uncountable Sets}
\begin{theorem}
    The set of real numbers $\{x:0\leq x<1\}=[0,1)$ is uncountable.
\end{theorem}
\begin{proof}[Cantor]
    Suppose it's countable, say $[0,1)=\{r_i:i\in\N\}$. Let $r_i=.r_{i1}r_{i2}\ldots$, with $r_{ij}\in\{0,\ldots,9\}$.
    Define $a$ by $a=.a_1a_2a_3\ldots$ were
    \[a_k=
    \begin{cases}
        1 &: r_{kk}\in\{5,6,7,8,9\}\\
        8 &: r_{kk}\in\{0,1,2,3,4\}
    \end{cases}
    \]
    and note that $a$ has a unique decimal representation. Since $a_k\neq r_{kk}$ for any $k$, $a\neq r_k$ for any $k$.
\end{proof}
\begin{remark}[Author's Remark]
    If you work with some topological properties, you can work with sets called \textit{perfect sets}. Perfect sets
    are closed sets that contain no isolated points: any element $a\in S$ can be written as a limit $\lim\{a_i\}$
    where $a_i\in S\setminus\{a\}$. In particular, the interval $[0,1]$ is a perfect set. We then have the following
    theorem:
\end{remark}
\begin{theorem}
    Non-empty perfect sets are uncountable.
\end{theorem}
\begin{proof}
    If $S$ is perfect, then $S$ is certainly not finite: given any $x\in S$, we can use increasingly small open
    neighbourhoods about $x$, all of which intersect $S\setminus\{x\}$ and avoid any previous elements of the sequence,
    thus constructing a countably infinite subset. Thus $S$ is either countable or uncountable. Suppose it were countable
    and write
    \[S=\{x_1,x_2,x_3,\ldots\}\]
    and consider the interval $U_1=\{x_1-1,x_1+1\}$. Now we construct inductively a sequence of nested intervals. Let
    $U_1\subset\ldots\subset U_k$ be previous intervals and $x_1,\ldots,x_k$ be previous points. Now choose $x_{k+1}\in U_k$
    and some neighbourhood $U_{k+1}$ so that $x_1,\ldots,x_k\notin U_{k+1}$ (this can be done since we only need to
    avoid finitely many points), and $\overline{U_{k+1}}\subset U_k$. But now we have a sequence $\{U_n\}$ of sets
    and $\{x_n\}$ of points so that
    \begin{enumerate}
        \item $x_k\in U_k$.
        \item $\overline{U_{k+1}}\subset U_k$
        \item $x_j\notin U_k$ for all $0<j<n$
    \end{enumerate}
    But now consider the set
    \[V=\bigcap_{n=1}^\infty \left( \overline{U_n}\cap S \right)\]
    Each set $\overline{U_N}\cap S$ is closed and bounded, hence compact, and $\overline{U_{n+1}}\cap S\subset\overline{U_{n}}\cap S$.
    Then by the nested compact set lemma, $V$ is non-empty and contains some element $v$. But $v\neq x_i$ for all $i$,
    since $v\in U_{i+1}$ but $x_i\notin U_{i+1}$. Thus our enumeration is incomplete, and $S$ is not countable.
\end{proof}
Note that the proof is essentially the diagonalization argument described above!
\begin{corollary}
    $\R$ is uncountable.
\end{corollary}
\begin{proof}
    Suppose $\R$ is countable, say $g:\R\to\N$ is a bjection. Then
    \[ g:[0,1)\subseteq\R\to\N\]
    so
    \[g\circ j:[0,1)\to\N\]
    is a bijection, so $[0,1)$ is countable - a contradiction.
\end{proof}
\begin{example}
    There exist transcendental numbers.
\end{example}
\begin{proof}
    The set of algebraic numbers is countable: there are a countable number of minimal polynomials, each of which has finitely many roots which are the algebraic numbers.
\end{proof}

\section{Cardinal Numbers}
We use the following notation: $|\N|=\aleph_0$, $|\R|=\aleph_1$. But does this notation make sense? This is the subject of
the Continuum Hypothesis: is there a set $A$ with $|\N|<|A|<|\R|$? This is undecidable; it is independent of the standard
axioms (ZFC axioms).
\begin{definition}
    Given a set $A$, the power set of $A$ denoted $\Ps(A)$ is defined as $\Ps(A)=\{x:x\subseteq A\}$.
\end{definition}
\begin{theorem}[Cantor]
    For any set $A$, $|A|<|\Ps(A)|$, where $|A|<|B|$ if $|A|\leq|B|$ and $|A|\neq|B|$.
\end{theorem}
\begin{proof}
    We certainly have an injection given by the map $a\mapsto\{a\}$, so $|A|\leq|\Ps(A)|$. Thus suppose we have
    some bijection $g:A\to\Ps(A)$. Define the set
    \[B=\{a\in A:a\notin g(a)\}\subseteq A\]
    Since $B\subseteq A$, we have $B\in\Ps(A)$. Hence there exists $x\in A$ such that $g(x)=B$. But now we have our contradiction
    in two cases! If $x\in B$, then $x\notin g(x)=B$. If $x\notin B=g(A)$, then $x\in B$. Thus no such $g$ exists.
\end{proof}
Using this we can construct an infinite list of cardinalities, since $|A|<|\Ps(A)|<|\Ps(\Ps(A))|<\cdots$.
\begin{definition}
    We define $2^A=\{f:A\to\{0,1\}\}$.
\end{definition}
For example, if $|A|=n$, then $|2^A|=2^n=|\Ps(A)|$.
\begin{theorem}
    $|2^A|=|\Ps(A)|$.
\end{theorem}
\begin{proof}
    Define $g:\Ps(A)\to 2^A$ by $B\mapsto \mathds{1}_B$ where $\mathds{1}_B$ is the indicator function defined as
    \[\mathds{1}_B=
    \begin{cases}
        0 &:x\notin B\\
        1 &:x\in B
    \end{cases}
    \]
    and $\mathds{1}_B\in 2^A$ certainly. $g$ is injective: if $B,C\subseteq A$ and $B\neq C$, then there exists
    some $x\in B$ but $x\notin C$ without loss of generality so $\mathds{1}_B(x)=1$ and $\mathds{1}_C(x)=0$. $g$
    is surjective: take $f\in 2^A$ and set $B=\{x\in A:f(x)=1\}$. Then $f=\mathds{1}_B$ so $g(B)=f$.
\end{proof}
\begin{corollary}
    $|A|<|2^A|$.
\end{corollary}
\begin{theorem}[Schroeder-Bernstein]
    If $|A|\leq|B|$ and $|B|\leq|A|$ then $|A|=|B|$.
\end{theorem}
\begin{proof}
    General idea: partition $A$ into two sections, $D$ and $D^c$ so that $D^c=g(f(D)^c$. If this holds, then we can define
    the bijection as
    \[\phi(x)=
    \begin{cases}
        f(x)&:x\in D\\
        g^{-1}(x)&:x\in D^c
    \end{cases}
    \]
    Define $Q:\mathcal{P}(A)\to\mathcal{P}(A)$ by the map
    \[E\mapsto\left[g(f(E)^c)\right]^c\subseteq A\]
    We wish to show that $Q$ has a fixed point, that is some $D\subseteq A$ such that $Q(D)=D$.

    We first show that if $E\subseteq F\subseteq A$, then $Q(E)\subseteq Q(F)$. This is simply a matter of following
    definitions.
    \begin{align*}
        f(E)\subseteq f(F)&\Rightarrow f(E)^c\supseteq f(F)^c\\
        &\Rightarrow g(f(E)^c)\subseteq g(f(F)^c)\\
        &\Rightarrow (g(f(E)^c))^c\subseteq (g(f(F)^c))^c\\
        &\Rightarrow Q(E)\subseteq Q(F)
    \end{align*}
    Now let $\mathcal{D}=\{E\subseteq A:E\subseteq Q(E)\}$. Set $D=\bigcup_{E\in\mathcal{D}}E\subseteq A$. If
    $E\in\mathcal{D}$, then $E\subseteq D$. By the claim, $Q(E)\subseteq Q(D)$. If $E\in\mathcal{D}$ then
    $E\subseteq Q(E)\subseteq Q(D)$, since $E\subseteq D$. So
    \begin{align*}
        \bigcup_{E\in\mathcal{D}}E\subseteq Q(D) &\Rightarrow Q(D)\subseteq Q(Q(D))\\
        &\Rightarrow Q(D)\in\mathcal{D}\\
        &\Rightarrow Q(D)\subseteq D
    \end{align*}
    Hence $D=Q(D)$.

\end{proof}
As discussed at the beginning, cardinality is an equivalence relation. The notation $|A|\leq|B|$ also makes sense as an
ordering by Schroeder-Bernstein. Finally by Cantor's argument, we have an infinite set of cardinalities.
\begin{corollary}\hspace{1cm}
    \begin{enumerate}
        \item If $A_1\subseteq A_2\subseteq A_3$, and $|A_1|=|A_3|$, then $|A_1|=|A_2|=|A_3|$.
        \item $|(0,1)|=|[0,1)|=|\R|$
        \item $|\R|=|2^{\N}|$.
    \end{enumerate}
\end{corollary}
\begin{proof}
    \begin{enumerate}
        \item We have injections $i,j$
            \[A_1\overset{i}{\hookrightarrow}A_2\overset{j}{\hookrightarrow}A_3\]
            given by the embedding maps, and a bijection $k:A_3\to A_1$. Then $k\circ j:A_2\to A_1$ is an injection, so by
            Schroeder-Bernstein, $|A_1|=|A_2|$ and $|A_2|=|A_3|$ by transitivity.
        \item It suffices to show $|(0,1)|=|\R|$. Consider $f(x)=\arctan x$ which is a bijection $f:\R\to\left( \frac{-\pi}{2},\frac{\pi}{2} \right)$.
            Thus
            \[\frac{1}{\pi}\arctan x+\frac{1}{2}:\R\to(0,1)\]
            is a bijection. There are many other examples of such functions! A good exercise is to find a rational function.

            \textit{Alternative Proof}:
            \begin{center}
                \begin{tikzpicture}[scale=2]
                    \draw (-3,0) -- (3,0);
                    \draw (0,-0.5) -- (0,1.5);
                    \draw[name path=semicirc] (1,1) arc (0:-180:1);
                    \draw[radius=0.05,fill=white] (1,1) node[above right]{$1$} circle;
                    \draw[radius=0.05,fill=white] (-1,1) node[above left]{$0$} circle;
                    \draw[radius=0.05,fill=black] (0,1);
                    \draw[dashed] (0,1) -- (2.5,0);
                    \draw[dashed, name path=line] (0,1) -- (-1.5,0)node[below]{$x$};
                    \path [name intersections={of=semicirc and line,by=E}];
                    \node [fill=red,inner sep=2pt,label=-90:$f(x)$] at (E) {};
                \end{tikzpicture}
            \end{center}
        \item $|\R|=|2^{\N}|$. Recall $2^{\N} =\{f:\N\to\{0,1\}\}$. Show $|[0,1)|=|2^{\N}|$. Take $r\in[0,1)$ and write as $r=.r_1r_2r_3\ldots$
            where $r_j\in\{0,1\}$ (binary representation of $r$. Define $f_r(n)=r_n,n\in\N$ so $f_r:\N\to\{0,1\}$ so $f_r\in 2^{\N}$.
            Define $i:[0,1)\to2^{\N}$ by the map $r\mapsto f_r$. This is injective since if $r\neq r'$, then the $k^{th}$ digits
            are different for some $k$ and that means $f_r\neq f_{r+1}$ and $|[0,1)|\leq|2^{\N}|$.

            Similarly, we have an injection $2^{\N}\to[0,1)$ given
            \[f\mapsto0.0f(1)0f(2)0f(3) \ldots \in [0,1)\]
            This is an injection because non-unique binary representation have to end with a tail of 1's (in one case) and
            a tail of 0's (in the other case). (A good exercise is to think about how to formalize this properly). Thus
            by Schroeder-Bernstein, the result follows.
    \end{enumerate}
\end{proof}
\begin{theorem}
    For any prime $p$, $c^p\equiv c\pmod{p}$.
\end{theorem}
\begin{proof}
    This follows by induction.
    For $c=0,1$ this is obvious, and if it holds for $c$, then by the binomial theorem $(c+1)^p=c^p+1=c+1\pmod{p}$.
\end{proof}
This generalizes to the Euler-Fermat Theorem:
\begin{theorem}
    If $(c,m)=1$ then $c^{\phi(m)}\equiv 1\pmod{m}$
\end{theorem}
\begin{proof}
    Note that $\phi(p^l)=p^l-p^{l-1}=p^l\left(1-\frac{1}{p}\right)$, and it can be shown that $\phi$ is multiplicative for coprime values, so
    \[\phi(n)=n\left(1-\frac{1}{p_1}\right)\cdots\left(1-\frac{1}{p_k}\right)\]
    where $p_1,\ldots,p_k$ are the prime divisors of $n$.
\end{proof}
Recall that $G(k)=\min t$ such that for all $n\geq n_0$, $n=x_1^k+\cdots+x_t^k$ for $x_i\geq 0$.
\begin{theorem}
    If $k>1$, then $g(k)\geq k+1$.
\end{theorem}
\begin{proof}
    We first show that $G(k)\geq k$.
    Suppose not, and get $n_0$ so that for all$ n\geq n_0$, $n=x_1^k+\cdots+x_{k-1}^k$.
    Fix $N$, and we get the number of integers with such a representation with $N\geq n_0$.
    Then $x_i^k\leq t\leq N$, so $0\leq x_i\leq\lfloor\sqrt[k]{N}\rfloor$.
    Thus the number of formal sums $x_1^k+\cdots+x_{k-1}^k$ denoted by $B$ must satisfy $B\geq N-n_0$.
    Furthermore, $B\sim N^{k/(k-1)}$ while $N-n_0\sim N$, a contradiction.

    We now show that $G(k)\geq k+1$.
    Assume not, so $\exists n_0$ so $\forall n>n_0$, $n=x_1^k+\cdots+x_k^l$ and let $A'$ denote the number of representable integers up to $N$, and $A'\geqN-n_0$.
    Now let $B'$ denote the number of formal sums quotiented by permutation.
    Thus $B'\geq A'$, where $A'\sim N$ but $B'\sim \frac{(\sqrt[k]{N})^k}{k!}=\frac{N}{k!}$, a contradiction.

    Let's compute $B'$ more precisely.
    We choose $k$ pieces from $0,1,\ldots,\lfloor\sqrt[k]{N}\rfloor$, where repetition is allowed.
    The number of ways to choose such $k$ pieces is given by the number of $(\lfloor\sqrt[k]{N}\rfloor+1)-$part compositions of $k$, so that
    \[B'=\binom{k+\lfloor\sqrt[k]{N}\rfloor}{k}=\frac{(\left\lfloor\sqrt[k]{N}\right\rfloor+k)\cdots+(\left\lfloor\sqrt[k]{N}\right\rfloor+1)}{k!}\]
    Since $B'\geq A'$, we have
    \[\frac{1}{k!}\left(1+\frac{k}{\sqrt[k]{N}}\right)\left(1+\frac{k-1}{\sqrt[k]{N}}\right)\cdots\left(1+\frac{1}{\sqrt[k]{N}}\right)\geq1=\frac{k_0}{N}\]
    and as $N\to\infty$, everything goes to $1$ except the first term.
\end{proof}
\end{document}
