\documentclass[12pt, a4paper]{book}
\usepackage[ascii]{inputenc}
\usepackage[left=2cm,right=2cm,top=2cm,bottom=4cm]{geometry}
\usepackage[protrusion=true,expansion=true]{microtype}

\usepackage{amsmath}
\usepackage{amsfonts}
\usepackage{amssymb}
\usepackage{tikz, pgfplots}
\usetikzlibrary{intersections}
\usepackage{kpfonts}
\usepackage{dsfont}
\pgfplotsset{compat=1.13}
\usepackage{emptypage}

\DeclareMathOperator{\N}{\mathbb{N}}
\DeclareMathOperator{\Q}{\mathbb{Q}}
\DeclareMathOperator{\Z}{\mathbb{Z}}
\DeclareMathOperator{\R}{\mathbb{R}}
\DeclareMathOperator{\C}{\mathbb{C}}
\DeclareMathOperator{\F}{\mathbb{F}}
\DeclareMathOperator{\GL}{GL}
\DeclareMathOperator{\sgn}{sgn}

\usepackage{graphicx}
\usepackage{enumitem}
\setenumerate{}

%-----------------------------------------------------------------------------------------------------------------
% Some fancy macros // May eventually move these into separate files or something and merge when building template
\renewcommand{\d}[1]{\ensuremath{\operatorname{d}\!{#1}}} % dx macro for integrals
\newcommand{\hess}[1]{\ensuremath{\operatorname{H}\!{#1}}} % Hessian
\newcommand{\diff}[1]{\ensuremath{\operatorname{D}\!{#1}}} % Jacobian
\newcommand{\inner}[2]{\left\langle #1, #2 \right\rangle} % inner product
\newcommand{\norm}[1]{\left\lVert#1\right\rVert} % norm
\newcommand{\cpl}[1]{\overline{#1}} % complement
\renewcommand{\v}[1]{\mathbf{#1}} % vector
\newenvironment{amatrix}[1]{% augumented matrix - make sure to have # columns less than required amount
  \left(\begin{array}{@{}*{#1}{c}|c@{}}
}{%
  \end{array}\right)
}
%-----------------------------------------------------------------------------------------------------------------
% Define theorem environments, along with a custom proof environment
\usepackage[thref, thmmarks,amsmath]{ntheorem}
\newcommand{\itref}[1]{\textit{\thref{#1}}}

\newtheorem{theorem}{Thm.}[section]
\newtheorem{lemma}[theorem]{Lemma}
\newtheorem{definition}[theorem]{Def'n.}
\newtheorem{corollary}[theorem]{Cor.}
\newtheorem{proposition}[theorem]{Prop.}

\theorembodyfont{\upshape}
\newtheorem{remark}[theorem]{Rmk.}
\newtheorem{exercise}[theorem]{Exc.}
\newtheorem{example}[theorem]{Ex.}
\theoremseparator{}
\theoremindent0.0cm
\theoremstyle{nonumberplain}
\theoremheaderfont{\scshape}
\theoremsymbol{$\square$}
\newtheorem{proof}{Proof}

%-----------------------------------------------------------------------------------------------------------------
% Define Document Variables
\newcommand{\assignmentname}{Course Notes}
\newcommand{\classname}{Introduction to Abstract Algebra}
\newcommand{\semester}{BSM Fall 2018}

% Define a title page for the document
%----------------------------------------------------------------------------------------------------------------------
% Define headings for each page
\usepackage{fancyheadings}
\pagestyle{fancy}
\lhead{Alex Rutar\\arutar@uwaterloo.ca}
\rhead{\classname: \assignmentname\\\semester}
\cfoot{\thepage}
\setlength{\headheight}{50pt}
%----------------------------------------------------------------------------------------------------------------------
\begin{document}
\pagenumbering{roman}
\begin{titlepage}
    \centering
    \vspace{5cm}
    {\huge\textbf{\assignmentname}\par} % Assignment Name
    \vspace{2cm}
    {\Large\textbf{\classname}\par} % Class
    \vspace{3cm}
    {\Large\textit{Alex Rutar}\par}

    \vfill

% Bottom of the page
    {\large \semester \par} % Due Date
\end{titlepage}
%----------------------------------------------------------------------------------------------------------------------
% \newpage\null\thispagestyle{empty}\textit{This page is left intentionally blank.}\newpage
\pagenumbering{roman}
\tableofcontents
\pagenumbering{arabic}
\chapter{Fundamentals of Groups}
\section{Basics of Groups}
\begin{definition}
    We say that $(G,*)$ with $*:G\times G\to G$ is a \textbf{group} if for all $a,b,c\in G$
    \begin{enumerate}
        \item $(a*b)*c=a*(b*c)$
        \item $\exists e\in G:\quad a*e=a=e*a$
        \item $\exists u\in G:\quad a*u=e=u*a$
    \end{enumerate}
\end{definition}
We have our first basic proposition:
\begin{proposition}
    The identity and inverses are unique.
\end{proposition}
\begin{proof}
    If $e,f$ are both identities, then $e=e*f=f$.
    If $u,v$ are both inverses of $x$, then $u*(x*v)=u*e=u$ and $(u*x)*v=e*v=v$ so $u=v$.
\end{proof}
\begin{definition}
    If $ab=ba$ for all $a,b\in G$ then we say that $G$ is \textbf{commutative}.
\end{definition}
\begin{definition}
    Let $G$ be a group with $G=\{g_1,g_2,\ldots,g_n\}$.
    Then the \textbf{Cayley Table} for $G$ is the matrix $M\in M_n(G)$ where $M_{ij}=g_ig_j$.
\end{definition}
\begin{proposition}
    In each column or row, each element occurs exactly once.
    Furthermore, if $M_{ij}=e$, then $M_{ji}=e$.
\end{proposition}
\begin{proof}
    This follows directly by left or right cancellation, and by commutativity of the elements with their inverse.
\end{proof}
\begin{definition}
    Let $(G,\Diamondblack)$, $(H,\star)$ be groups.
    A mapping $f:G\to H$ is called an \textbf{homomorphism} if
    \[f(u\Diamondblack v)=f(u)\star f(v)\]
    If $f$ is also a a bijection, then we call $f$ an \textbf{isomorphism}.
\end{definition}
\begin{proposition}
    $G$ and $H$ are isomorphic if and only if their Cayley Tables are the same up to permutation of elements.
\end{proposition}
\begin{proof}
    Obvious.
\end{proof}
\subsection{The group $\Z_m$}
\begin{definition}
    Let $\sim$ be an equivalence relation.
    We then define the \textbf{quotient group} $G/\sim$ given by the equivalence classes of elements in $G$.
\end{definition}
To construct $\Z_m$, we define $\Z_m=\Z/\sim$ where $a\sim b$ if $a\cong b\pmod{m}$.
Since we have a division algorithm in $\Z$, for any $d\in\Z$, we can write $d=tm+r$ with $0\leq r\leq m-1$.
Thus $\overline{d}=\overline{r}$, so we can represent $\Z_m=\{\overline{0},\overline{1},\ldots,\overline{m-1}\}$.
As a result we usually do not bother writing $\overline{\cdot}$.
\begin{proposition}
    We have $\overline{a}+\overline{b}=\overline{a+b}$ and $\overline{a}\cdot\overline{b}=\overline{ab}$.
\end{proposition}
\begin{proof}
    Obvious.
\end{proof}
\begin{theorem}
    $\Z_m^\times=\{\overline{a}\mid\gcd(a,m)=1\}$.
\end{theorem}
\begin{proof}
    Assume $\overline{a}\in\Z^\times_m$ so there exists $\overline{x}$ with $\overline{x}\cdot\overline{a}=1$.
    Then $\overline{xa}=\overline{1}$ so $xa\cong 1\pmod{m}$ so $m|xa-1$.
    Let $d=\gcd(a,m)$ so $d|a$ and $d|m$.
    Thus $d|xa-1$ and $d|xa$ so $d|1$ and $\gcd(a,m)=1$.

    Conversely, suppose $\gcd(a,m)=1$.
    Then by B\'ezout's Lemma, get $x,y$ so that $xa+ym=1$, so $xa\cong 1\pmod{1}$ and $\overline{xa}=\overline{1}$ and $\overline{x}\overline{a}=\overline{1}$ and we have our multiplicative inverse.
\end{proof}
We thus have $|\Z_m^\times|=\phi(m)$.
\section{Subgroups}
\begin{definition}
    A subset $H$ of a group $G$ is called a \textbf{subgroup} if $H$ is also a group with the same operation.
    We write $H\leq G$.
\end{definition}
For example, $(\Z,+)\leq(\Q,+)\leq(\R,+)\leq(\C,+)$.
Note that associativity automatically holds since every element of $H$ is an element of $G$.
Furthermore, $1_H=1_G$ since $1_H1_G=1_H=1_H1_H$ where the first equality holds since $1_G$ is an identity, and the second since $1_H$ is an identity.
As a result, inverses in $H$ are inverses in $G$.
\subsection{Subgroup Tests}
\begin{proposition}[First Subgroup Test]
    A subset $H$ of a group $G$ is a subgroup if and only if
    \begin{enumerate}
        \item $H\neq\emptyset$
        \item $x,y\in H\Rightarrow xy\in H$
        \item $x\in H\Rightarrow x^{-1}\in H$
    \end{enumerate}
\end{proposition}
\begin{proof}
    Follows by above discussion.
\end{proof}
\begin{proposition}[Second Subgroup Test]
    A subset $H$ of a group $G$ is a subgroup
    \begin{enumerate}
        \item $H\neq\emptyset$
        \item $x,y\in H\Rightarrow xy^{-1}\in H$
    \end{enumerate}
\end{proposition}
That the first subgroup test implies the second is obvious.
Coversely, the identity is in $H$ since $xx^{-1}\in H$.
Thus get closure under inversion by choosing $x$ as the identity to get inverses.
Then if $x,y\in H$, $x,y^{-1}\in H$ so $x(y^{-1})^{-1}=xy\in H$.

Furthermore, if $G$ is finite, it suffices to show closure under multiplication, since inverses can be optained by repeated multiplication.
\begin{proposition}
    Arbitrary intersections of subgroups are also subgroups.
\end{proposition}
\begin{proof}
    Obvious.
\end{proof}
\subsection{Cosets of Subgroups}
\begin{definition}
    Let $H\leq G$, $g\in G$.
    Then the \textbf{right coset} of $H$ by $g$ is the set $Hg:=\{hg:h\in H\}$.
    Similarly, the \textbf{left coset} of $H$ by $g$ is the set $gH:=\{gh:h\in H\}$.
\end{definition}
\begin{example}
    Consider $G=\Z_{13}^\times=\{1,2,\ldots,12\}$ and $H=\langle 3\rangle=\{1,3,9\}$.
    Then the cosets of $H$ are given by
    \begin{align*}
        H1 &= \{1,3,9\} & H2 &= \{2,5,6\}\\
        H3 &= H1 & H4 &= \{4,10,12\}\\
        H5 &= H2 & H6 &= H2\\
        H7 &= \{7,8,11\} & H8 &= H7\\
        H9 &= H1 & H10 &= H4\\
        H11 &= H7 & H12 &= H4
    \end{align*}
    so there are 4 disjoint cosets of $H$.
\end{example}
This inspires the following theorem:
\begin{theorem}
    Let $H\leq G$.
    Then
    \begin{enumerate}[nolistsep]
        \item $|Hg|=|H|$
        \item $Hg=H\Leftrightarrow g\in H$
        \item For any $x,y\in G$, either $Hx=Hy$ or $Hx\cap Hy=\emptyset$
        \item $Hx=Hy\Leftrightarrow xy^{-1}\in H$
    \end{enumerate}
\end{theorem}
\begin{proof}
    \begin{enumerate}
        \item The map $\cdot g:H\to Hg$ is bijective since it has an inverse.
        \item This is a special case of (4) with $x=g$, $y=1$.
        \item Suppose $Hx\cap Hy\neq\emptyset$.
            Thus let $z\in Hx\cap Hy$ and write $z=h_1x=h_2y$.
            Then for any $hx\in Hx$, $hx=hh_1^{-1}h_1x=hh_1^{-1}h_2y\in Hy$ so $Hx\subseteq Hy$.
            The identical argument works in reverse, so equality holds.
        \item Assume $Hx=Hy$, and if $x\in Hx$, then $x\in Hy$ so $x=hy$ and $xy^{-1}=h$.
            Conversely, suppose $xy^{-1}\in H$, then $xy^{-1}y\in Hy$ so $x\in Hy$.
            Also, $x\in Hx$ so $x\in Hx\cap Hy\neq\emptyset$ so by (3), $Hx=Hy$.
    \end{enumerate}
\end{proof}
\begin{definition}
    The \textbf{index} of a subgroup $H$ in a group $G$ is denoted $|G:H|$ and denotes the number of distinct right cosets of $H$.
\end{definition}
\begin{proposition}
    $Hx\mapsto x^{-1}H$ is a one-to-one correspondence between right cosets and left cosets.
\end{proposition}
Thus $G$ is a disjoint union of $|G:H|$ right cosets of $H$, each of size $|H|$.
Therefore we have
\begin{corollary}
    $|G|=|G:H|\cdot|H|$
\end{corollary}
\begin{theorem}[Lagrange]
    Suppose $G$ is a finite group.
    Then
    \begin{enumerate}
        \item For any $H\leq G$, $|H|\mid|G|$.
        \item For any $g\in G$, $o(g)||G|$.
    \end{enumerate}
\end{theorem}
\begin{proof}
    \begin{enumerate}
        \item Since $|G|=|G:H|\cdot|H|$, $|G:H|$ is a positive integer.
        \item $o(g)=|\langle g\rangle|$ and it follows by (1).
    \end{enumerate}
\end{proof}
\subsection{Center of a Group}
\begin{definition}
    For any $g\in G$, define
    \[C_G(g)=\{x\in G:gx=xg\}\]
    the \textbf{centralizer} of $g$ in $G$.
    Then define the \textbf{center} of a group $G$
    \[Z(G)=\bigcap_{g\in G}C_G(g)\leq G\]
\end{definition}
Note that the center of a group is the set of elements which commute with everything in the group.
These are indeed groups:
We certainly have $1\in C_G(g)$.
Also, if $x,y\in G$, then $gx=xg$ and $gy=yg$ so that $gxy=xgy=xyg$.
If $x\in C_G(g)$, then $gx=xg$ so $g=xgx^{-1}$ and $x^{-1}g=gx^{-1}$.
\subsection{Conjugacy Classes}
This definition inspires the following definition:
\begin{definition}
    We say that $f$ is a \textbf{conjugate} of $g$ if and only if there exists $x\in G$ such that $x^{-1}gx=f$.
\end{definition}
Denote the binary relation by $\sim$: we will show that this is an equivalence relation:
\begin{enumerate}[nolistsep]
    \item Reflexive: $g\sim g$ by $x=1$
    \item Symmetric: If $g\sim f$, then $x^{-1}gx=f$ so $g=xfx^{-1}=(x^{-1})^{-1}fx^{-1}$
    \item Transitive: If $f\sim g$ and $g\sim h$, get $x,y$ so $x^{-1}gx=f$ and $y^{-1}fy=h$ so
        \[h=y^{-1}x^{-1}gxy=(xy)^{-1}g(xy)\]
\end{enumerate}
\begin{definition}
    These equivalence classes are called the \textbf{conjugacy classes} of $G$.
\end{definition}
We denote the conjugacy class of $g\in G$ by $C_g=\{x^{-1}gx:x\in G\}$.
Note that $|C_g|=1$ if and only if $C_g=\{g\}$ if and only if $x^{-1}gx=g$ for any $x\in G$ if and only if $gx=xg$ and $g\in Z(G)$.
\begin{theorem}
    For any $g\in G$, $|C_g|\cdot|C_G(g)|=|G|$.
\end{theorem}
\begin{proof}
    Consider $\alpha:\{\text{Right cosets of $D_G(g)$}\}\longrightarrow C_g$ defined by $C_G(g)\cdot x\mapsto x^{-1}gx$.
    This is well defined and injective:
    \begin{align*}
        C_G(g)x=C_G(g)y &\Leftrightarrow xy^{-1}\in C_G(g)\\
                        &\Leftrightarrow g(xy^{-1})\\
                        &\Leftrightarrow (xy^{-1})g
    \end{align*}
    so it suffices to show the map is surjective.
    In fact, any element of $C_g$ is of the form $x^{-1}gx=\alpha(C_G(g)x)$.
    Thus $\alpha$ is bijective, so $|G:C_G(x)|=|C_g|$ and
    \[|G|=|G:C_G(g)|\cdot|C_G(g)|=|C_g|\cdot|C_G(g)|\]
\end{proof}
\begin{corollary}
    If $G$ is finite, $g\in G$, then $|C_g|\mid|G|$.
\end{corollary}
We have the following nice application:
\begin{theorem}
    If $|G|=p^2$ for $p$ prime, then $G$ is commutative.
\end{theorem}
\begin{proof}
    For any $g\in G$, $|C_g|\mid|G|=p^2$ so $|C_g|$ there are three cases.
    Note that $|C_g|=p^2$ is impossible, since $C_1=\{1\}$ and the remainder has fewer elements.
    Thus let $a$ denote the number of conjugacy classes of size $1$ by $a$, and the number of conjugacy classes of size $p$ by $b$.
    Since $G$ is a disjoint union of conjugacy classes, we have $|G|=p^2=a+bp$ so that $p|a$.
    Furthermore, $a\neq 0$ since $|C_1|=1$, so $a\geq p$.
    Furthermore, $|C_g|=1$ if and only if $g\in Z(G)$, so $a=|Z(G)|\geq p$.
    Since $Z(G)\leq G$, by Lagrance, $|Z(G)|\mid|G|=p^2$, so $|Z(G)|=p$ or $|Z(G)|=p^2$.
    If $|Z(G)|=p$, pick any $x\in G$ with $x\notin Z(G)$ and consider $C_G(x)$.
    Since $Z(G)\leq C_G(x)$, we must have $p+1\leq |C_G(x)|$ and $|C_G(x)|=p^2$ so $C_G(x)=G$ and $x\in Z(G)$, a contradiction.
    Thus $Z(G)=p^2$ and the group is commutative.
\end{proof}
Note that if $|G|=p$ prime, then $G$ is cyclic.
Since $o(g)||G|=p$, and $o(g)\neq 1$ if $g\neq 1$; we must have $o(g)=p$ and $\langle g\rangle=G$.

Now if $H\leq G$, then $x^{-1}Hx=\{x^{-1}hx:h\in H\}\leq G$, as can be verified.
\begin{definition}
    A subgroup $K$ of $G$ is \textbf{conjugate} to $H$ in $G$ if and only if there exists $x\in G$ with $x^{-1}Hx=K$.
    We write $H\sim K$, and the equivalence classes are called \textbf{conjugacy classes} of subgroups.
\end{definition}
\begin{theorem}
    \begin{enumerate}[nolistsep]
        \item Conjugate elements are of the same order.
        \item Conjugate subgroups are isomorphic.
    \end{enumerate}
\end{theorem}
\begin{proof}
    \begin{enumerate}
        \item We have
            \begin{align*}
                (x^{-1}gx)^k=1 &\Leftrightarrow (x^{-1}gx)(x^{-1}gx)\cdots(x^{-1}gx)=1\\
                               &\Leftrightarrow x^{-1}g^Kx=1\\
                               &\Leftrightarrow g^kx=x\\
                               &\Leftrightarrow g^k=1
            \end{align*}
        \item I claim that the map $\alpha:H\to x^{-1}Hx$ by $h\mapsto x^{-1}hx$ is an isomorphism.
            We have $\alpha(h_1h_2)=x^{-1}h_1h_2x=x^{-1}h)1xx^{-1}h_2x=\alpha(h_1)\alpha(h_2)$, and bijectivity can be verified easily.
    \end{enumerate}
\end{proof}
For any group $G$, we always have $C_\{1\}=\{\{1\}\}$ and $C_G=\{G\}$.
A particularly nice type of conjugacy class are the ones with only 1 element.
We have
\[|C_H|=1\Leftrightarrow C_H=\{H\}\Leftrightarrow x^{-1}Hx=H(\forall x\in G)\Leftrightarrow Hx=xH(\forall x\in G)\]
\begin{definition}
    A subgroup $H$ which satisfies $Hx=xH$ for all $x\in G$ is called a \textbf{normal} subgroup.
    We say $H\triangleleft G$.
\end{definition}
\begin{definition}
    The \textbf{centralizer} of a subgroup $H$ in $G$ is
    \[C_G(H)=\{x\in G:hx=xh(\forall h\in H)\}=\bigcap\limits_{h\in H}C_G(h)\leq G\]
\end{definition}
Note that intersections of subgroups are subgroups.
\begin{definition}
    The \textbf{normalizer} of a subgroup $H$ in $G$ is
    \[N_G(H)=\{x\in G:Hx=xH\}=\{x\in G:x^{-1}Hx=H\}\leq G\]
\end{definition}
It is easy to verify this is a subgroup.
We thus have $H\triangleleft G$ if and only if $N_G(H)=G$.
We have some properties:
\begin{proposition}
    \begin{enumerate}
        \item $C_G(G)\leq N_G(H)$.
            In general, equality does not hold.
        \item $H\leq N_G(H)$.
        \item $H\leq C_G(H)$ iff $H$ is commutative.
        \item $N_G(H)=G$ iff $H$ is normal.
        \item $C_G(H)=G$ iff $H\leq Z(G)$.
    \end{enumerate}
\end{proposition}
\begin{example}
    Let $G=D_4$, $H=\langle r\rangle$.
    Then $s\in N_6(H)$ but $s\notin C_G(H)$.
\end{example}
\begin{proposition}
    A subgroup $H$ in $G$ is normal if and only if
    \begin{enumerate}
        \item $Hx=xH$ for all $x\in G$.
        \item $x^{-1}Hx=H$ for all $x\in G$.
        \item $N_G(H)=G$.
        \item For any $h\in H$, $x\in G$, $x^{-1}hx\in H$.
        \item $H$ is a union of some conjugacy classes.
    \end{enumerate}
\end{proposition}
\begin{proof}
    We only see $(4)\Leftrightarrow (5)$.
    We have
    \[\forall h\in H\forall x\in G x^{-1}hx\in H\Leftrightarrow\forall h\in H C_h\subseteq H\]
    which means that all conjugacy classes are either disjoint from $H$, or in $H$.
\end{proof}
We will most commonly use condition (4) to check normality.
\begin{example}
    For example, fix $G=GL_n(\R)$, so $SL_n(\R)=\{A\in M_n(\R):\det(A)=1\}$.
    This is indeed a subgroup: let's also verify that it is a normal subgroup.
    Also, if $h\in SL_n(\R)$ and $x\in GL_n(\R)$, then $\det(x^{-1}hx)=\det(x^{-1})\det(h)\det(x)=\det(h)=1$ so $x^{-1}hx\in SL_n(\R)$.
\end{example}
Why are normal subgroups nice?
If $H\triangleleft G$, and $x,y\in G$, then $(Hx)(Hy)=Hxy$.
We thus have an operation on cosets of $H$.
Furthermore, this action satisfies the properties of the group.
Thus $\{Hx:x\in G\}$ with the operation $HxHy=Hxy$ is a group, called the factor group or quotient group of $G$ by $H$.
\begin{example}
    Consider $G=\Z^\times_{13}$, $H=\langle 3\rangle$.
    Then $H2=\{256\}$, $H4=\{4,10,12\}$, $H7=\{7,8,11\}$.
    We have
    \begin{tabular}{c|c|c|c|c|}
        &H&H2&H4&H7\\
        \hline
        H &H&H2&H4&H7\\
        H2&H2&H4&H7&H\\
        H4&H4&H7&H&H2\\
        H7&H7&H&H2&H4
    \end{tabular}

\section{Examples of Finite Groups}
\subsection{Cyclic Groups}
\begin{definition}
    The \textbf{order of an element} $g\in G$ is $o(g):=\left\lvert\{g^d|d\in\Z\}\right\rvert$.
    The \textbf{order of a group} $G$ is $|G|$.
\end{definition}
We certainly have $o(g)\leq|G|$ for any $g\in G$.
Equality holds when $o(g)=\infty$ and $G$ is countable, or $G=\{g^d:d\in\Z\}$.
\begin{definition}
    A collection $H=\{g_1,g_2,\ldots,g_k\}$ \textbf{generates} $G$ if we can write any $g\in G$ as a product of elements in $H$.
\end{definition}
\begin{definition}
    We say that $G$ is \textbf{cyclic} if $G=\{g^d:d\in\Z\}$ for some $g\in G$.
    Equivalently, it is generated by a set of cardinality one.
\end{definition}
\begin{example}
    Note that $\Z_{13}^\times$ is cyclic with generator $2$.
\end{example}
\begin{lemma}
    If $o(g)$ is finite and $d\in\Z$, then
    \[o(g^d)=\frac{o(g)}{\gcd(o(g),d)}\]
\end{lemma}
\begin{proof}
    Let $o(g)=K$ and $t=\gcd(K,d)$ and write $K=tK_1$ and $d=td_1$ with $K_1,d_1$ coprime.
    Thus $o(g^d)$ is the smallest positive integer $l$ with $(g^d)^l=1$.
    But then $(g^d)^l=1\Leftrightarrow g^{dl}=1\Leftrightarrow o(g)|dl$ and $k|dl$, that is $tK_1|td_1l$ and $k_1|d_1l$.
    Thus $K_1|l$ so the smallest positive intger $l$ is $K_1$ and $o(g^d)=K_1=\frac{o(g)}{\gcd(o(g),d)}$ as desired.
\end{proof}
\subsubsection{Subgroups of Cyclic Groups}
\begin{theorem}
    Any subgroup of a cyclic group is also cyclic.
\end{theorem}
\begin{proof}
    Let $G=\langle g\rangle$ be a cyclic group, $H\leq G$.
    If $H=\{1\}$, then $H=\langle 1\rangle$ is cyclic.
    Otherwise, there exists some $0\neq m\in\Z$ with $g^m\in H$.
    Now, there exists a smallest positive integer $k$ with $g^k\in H$.
    We see that $H=\langle g^k\rangle$.
    The reverse inclusion is obvious since $(g^k)^t\in H$ for all $t\in\Z$.
    For the forward inclusion, pick $x\in H$ so $x=g^d$ for some $d$.
    Then division with remainder yields $d=tk+r$ with $0\leq r\leq k-1$ so that
    $g^d=g^{tk+r}$ and $x=(g^k)^tg^r$ so $g^r=x(g^k)^{-t}\in H$.
    Minimality of $k$ forces $r=0$, so $d=tk$, $x=g^d=(g^k)^t\in\langle g^k\rangle$.
\end{proof}
If $|G|=o(g)=n$ finite, write $n=tk+r$, for $0\leq r\leq k-1$.
Then $g^r=g^n(g^k)^{-t}=(g^k)^{-t}\in H$, and again $r=0$, $n=tk$, $k|n$.

Now suppose $G=\langle g\rangle$ with finite order $n$.
Then $G=\{1,g,g^2,\ldots,g^{n-1}\}$, and subgroups of $G$ correspond to positive diviors of $n$.
Then $k|n\leftrightarrow\langle g^k\rangle=\{1,g^k,g^{2k},\ldots,g^{n-k}\}$
Now suppose $G=\langle g\rangle$ is infinite, and $G=\{\ldots,g^{-1},1,g,g^2,\ldots\}$.
Then subgroups of $G$ correspond to nonnegative integers, and $k\geq 0\leftrightarrow \langle g^k\rangle=\{\ldots,g^{-k},1,g^k,g^{2k},\ldots\}$.
\begin{example}
    Consider $G=\Z_{13}^\times=\left\langle 2\right\rangle$, $|\Z_{13}^\times|=12=o(2)$.
    \begin{center}
        \begin{tabular}{c|c}
            Divisor of 12&Subgroup of $\Z_{13}^\times$\\
            \hline
            1& $\langle 2^1\rangle=\langle 2\rangle=\Z_{13}^\times$\\
            1& $\langle 2^2\rangle=\langle 4\rangle=\{1,4,3,12,9,10\}$\\
            1& $\langle 2^3\rangle=\langle 8\rangle=\{1,8,12,5\}$\\
            1& $\langle 2^4\rangle=\langle 3\rangle=\{1,3,9\}$\\
            1& $\langle 2^6\rangle=\langle 12\rangle=\{1,12\}$\\
            1& $\langle 2^{12}\rangle=\langle 1\rangle =\{1\}$
        \end{tabular}
    \end{center}
\end{example}
\subsection{Permutation Groups}
Recall that $S_n$ is the symmetric group of degree $n$, consisting of all permutations of $[n]$.
Thus $|S_n|=n!$.
Instead of using the matrix form, we can write the permutation group using the cycle form.
\begin{example}
    Write
    \[f=\begin{pmatrix}1&2&3&4&5&6&7&8&9\\4&7&3&1&2&9&8&5&6\end{pmatrix}=(14)(2785)(3)(69)\]
    We can also write $(14)(2785)(69)$, in other words excluding elements which map to themselves.
\end{example}
In general, a cycle $(a_1a_2\ldots a_k)$ indicates that $a_1f=a_2$, $a_2f=a_3$,\dots,$a_kf=a_1$.
In $S_n$, each permutation can be expressed in a cycle form (using disjoint cycles).
The cycle form is unique up to ordering within the cycles, and ordering among the cycles.
\begin{example}
    In $S_5$, the possible cycle structures are
    \[I,(ab),(abc),(abcd),(abcde),(ab)(cd),(ab)(cde)\]
    We then have
    \begin{align*}
        o(I) &= 1\\
        o((ab)) &= 2\\
        o((abc)) &= 3\\
        o((abcd)) &= 4\\
        o((abcde)) &= 5\\
        o((ab)(cd)) &= 2\\
        o((ab)(cde)) &= 6
    \end{align*}
    For $f=(abc)$, $f^2=(abc)(abc)=(acb)$, $f^3=(abc)(acb)=abc$.
    For $f=(abcd)$, $f^2=(ac)(bd)$, $f^3=(abdc)(ac)(bd)(adcb)$, and $f^4=(abcd)(adcb)=(abcd)$.

    If $f=(a_1a_2\ldots a_k)$, $o(f)=k$.
\end{example}
\begin{proposition}
    Suppose $f=\gamma_1\gamma_2\ldots\gamma_i$ for disjoint cycles.
    Then $o(f)=lcm(o(\gamma_1),o(\gamma_2),\ldots,o(\gamma_i))$.
\end{proposition}
\begin{proof}
    Note that the $\gamma_i$ commute, so that
    \begin{align*}
        f^d=I &\Leftrightarrow (\gamma_1\gamma_2\ldots\gamma_i)^d=I\\
              &\Leftrightarrow \gamma_1^d\gamma_2^d\ldots\gamma_i^d=I\\
              &\Leftrightarrow \gamma_i^d=I\quad\forall i
    \end{align*}
    The last line holds since the $\gamma_i^d$ operates on disjoint sets.
    Thus we have our formula, as desired.
\end{proof}
Note that any finite permutation of $f\in S_n$ can be expressed as a composition of 2-cycles.
For example, $(abc)=(ab)(ac)$ and in general $(a_1a_2\ldots a_k)=(a_1a_2)(a_1a_3)\ldots(a_1a_k)$.
In general, any $k-$cycle can be replaced by a composition of $(k-1)$ 2-cycles.
This motivates the following definition:
\begin{definition}
    A permutation $f\in S_n$ is \textbf{even} if it can be expressed as a composition of an even number of 2-cycles.
    Then $f\in S_n$ is \textbf{odd} if it can be expressed as a composition of an odd number of 2-cycles.
\end{definition}
For example, $(15362)(4798)=(15)(13)(16)(12)(47)(49)(48)$ can be written as a composition of 7 2-cycles.
This is certainly not unique: for example $(26)=(21)(16)(21)$.
\begin{lemma}
    The identity permutation is not odd.
\end{lemma}
\begin{proof}
    For contradiction, assume
    \[I=\alpha_1\alpha_2\ldots\alpha_{k}\]
    and assume that such an odd $k$ is a minimal counterxample.
    We certainly have $k\geq 3$.
    Say $\alpha_1=(cd)$, so $c$ must be involved in another $\alpha_i$, or $d$ is mapped to $c$.
    Let $\alpha_r$ be the last 2-cycle involving $c$, say $\alpha_r=(cx)$.
    Now we rewrite $\alpha_{r-1}$ without changing $\alpha_{r-1}\alpha_r$.
    \begin{enumerate}[nolistsep]
        \item If $\alpha_{r-1}=(yz)$ disjoint from $\alpha_r=(cx)$, then $(yz)(cx)=(cx)(yz)$.
        \item If $\alpha_{r-1}=(cy)$ with $y\neq x$, then $(cy)(cx)=(xc)(xy)$.
        \item If $\alpha_{r-1}=(xy)$, $y\neq c$, then $(xy)(cx)=(yc)(yx)$.
        \item $\alpha_{r-1}=\alpha_r$ so $(cx)(cx)=I$, contradicting minimality.
    \end{enumerate}
    We can repeat this process until the last 2-cycle involving $c$ is $\alpha_1$, a contradiction.
\end{proof}
\begin{proposition}
    A permutation cannot be both even and odd.
\end{proposition}
\begin{proof}
    Suppose $f$ can be written as an even and odd permutation:
    \begin{align*}
        f &= \alpha_1\alpha_2\ldots\alpha_m\\
        f &= \beta_1\beta_2\ldots\beta_n\\
    \end{align*}
    but then
    \[I=\alpha_1\alpha_2\ldots\alpha_m\alpha_m\ldots\alpha_2\alpha_1=\beta_1\beta_2\ldots\beta_n\alpha_m\alpha_{m-1}\ldots\alpha_1\]
    so $I$ is odd, a contradiction.
\end{proof}
\begin{definition}
    We define the \textbf{signature} $\sgn(f)$ to be $1$ of $f$ is even, and $-1$ if $f$ is odd.
\end{definition}
\begin{proposition}
    \begin{enumerate}[nolistsep]
        \item $\sgn(f^{-1})=\sgn(f)$
        \item $\sgn(fg)=\sgn(f)\sgn(g)$
    \end{enumerate}
\end{proposition}
\begin{proof}
    Follows directly from the 2-cycle decomposition.
\end{proof}
\begin{definition}
    The \textbf{alternating group} of degree $n$ is the group $A_n=\{f\in S_n:\sgn(f)=1\}\leq S_n$.
\end{definition}
\begin{theorem}
    $|A_n|=\frac{n!}{2}$.
\end{theorem}
\begin{proof}
    We see two separate proofs.
    \begin{enumerate}
        \item Consider $\phi:A_n\to S_n\setminus A_n$ by $f\mapsto f(12)$.
            This is injective since if $\phi(f)=\phi(g)$, then $f(12)=g(12)$ and $f=g$.
            It is surjective: if $g$ is odd, then $g(12)$ is even that $\phi(g(12))=g$.
            Thus $\phi$ is bijective and $|A_n|=|S\setminus A_n|=|A|-|A_n|$ so $|A_n|=|S_n|/2=n!/2$.
        \item We claim that $|S_n:A_n|=2$.
            For $f\in S_n$ even, $f\in A_n$ so $A_nf=A_n$.
            For $f\in S_n$ odd, $f^{-1}$ is odd and $(12)f^{-1}$ is even and $(12)f^{-1}\in A_n$.
            Thus $A_n(12)=A_nf$, so there are only two cosets of $A_n$: $A_n$ and $A_n(12)$, and the result follows by Lagrange's Theorem.
    \end{enumerate}
\end{proof}
\subsubsection{Centralizers of Permutation Groups}
\begin{example}
    Consider $g=(12)(34)\in S_4$.
    Then
    \[C_{S_4}(g)=\{x\in S_4\mid gx=xg\}=\{I,(12)(34),(12),(34),(14)(23),(1324),(1423)\}\]
\end{example}
The key idea is to observe that $x^{-1}gx=g$, which is called the conjugate of $g$ by $x$.
\begin{example}
    Consider $f=(34)(1572)(86)(9)$, $g=(194)(368)(257)$.
    \begin{align*}
        g^{-1}fg &= (752)(863)(491)(34)(1572)(86)(194)(368)(257)\\
                 &= (16)(2597)(38)(4)\\
                 &= (3g)(4g)(1g 5g 7g 2g)(8g 6g) (9g)
    \end{align*}
\end{example}
In general, if $f,g\in S_n$ and $(a_1a_2,\ldots,a_k)$ is a cycle in the cycle form of $f$, then $(a_1z\,a_2z\ldots a_kz)$ is a cycle in the cycle form of $z^{-1}fz$.
To see this, $a_1z(z^{-1}fz)=a_1fz=a_2z$, so $a_1z$ maps to $a_2z$, and similarly for all the pairs of elements in the cycle.

If we now return to $(12)(34)x=x(12)(34)$, we have $x^{-1}(12)(34)x=(12)(34)$ so
\[(1x\,2x)(3x\,4x)=(12)(34)\]
Since the cycle form is unique up to rearranging within cycles, we have
\begin{center}
    \begin{tabular}{c|cccc|c}
        LHS & $1x$ & $2x$ & $3x$ & $4x$ & $x$\\
        \hline
        $(12)(34)$ & 1&2&3&4& $I$\\
        $(21)(34)$ & 2&1&3&4& $(12)$\\
        $(12)(43)$ & 1&2&4&3& $(34)$\\
        $(21)(43)$ & 2&1&4&3& $(12)(34)$\\
        $(34)(12)$ & 3&4&1&2& $(13)(24)$\\
        $(34)(21)$ & 3&4&2&1& $(1324)$\\
        $(43)(12)$ & 4&3&1&2& $(1423)$\\
        $(43)(21)$ & 4&3&2&1& $(14)(23)$
    \end{tabular}
\end{center}
Let's now compute the conjugacy classes of $S_n$.
Let's do $S_3$ first:
The conjugacy classes are given by
\[\{1\},\{(12),(13),(23)\},\{(123)\}\]
In general, the conjugacy classes in $S_n$ correspond to the possible cycle structures in $S_n$.
\end{example}
\subsection{Dihedral Groups}
Fix a regular polygon with $n$ vertices.
Let $D_n$ be the collection of rigid motions with map the regular $n-$polygon to itself.
Since $r^n=1$ and $s^2=1$, we have
\[D_n=\{1,r,r^2,\ldots,r^{n-1},s,sr,sr^2,\ldots,sr^{n-1}\}\]
Thus $|D_n|=2n$.
We can compute the oprations on $D_n$:
\begin{align*}
    r^a\cdot r^b &= r^{a+b}\\
    sr^a\cdot r^b &= sr^{a+b}\\
    r^a\cdot sr^b &= sr^{b-a}\\
    sr^a\cdot sr^b &= r^{b-a}
\end{align*}
Thus $o(sr^a)=2$ and $o(r^a)$ is given by the usual formula.
\end{document}
