\documentclass[12pt, a4paper]{book}
\usepackage[ascii]{inputenc}
\usepackage[left=2cm,right=2cm,top=2cm,bottom=4cm]{geometry}
\usepackage[protrusion=true,expansion=true]{microtype}

\usepackage{amsmath}
\usepackage{amsfonts}
\usepackage{amssymb}
\usepackage{tikz, pgfplots}
\usetikzlibrary{intersections}
\usepackage{kpfonts}
\usepackage{dsfont}
\pgfplotsset{compat=1.13}
\usepackage{emptypage}

\DeclareMathOperator{\N}{\mathbb{N}}
\DeclareMathOperator{\Q}{\mathbb{Q}}
\DeclareMathOperator{\Z}{\mathbb{Z}}
\DeclareMathOperator{\R}{\mathbb{R}}
\DeclareMathOperator{\C}{\mathbb{C}}
\DeclareMathOperator{\F}{\mathbb{F}}
\DeclareMathOperator{\GL}{GL}

\usepackage{graphicx}
\usepackage{enumitem}
\setenumerate{}

%-----------------------------------------------------------------------------------------------------------------
% Some fancy macros // May eventually move these into separate files or something and merge when building template
\renewcommand{\d}[1]{\ensuremath{\operatorname{d}\!{#1}}} % dx macro for integrals
\newcommand{\hess}[1]{\ensuremath{\operatorname{H}\!{#1}}} % Hessian
\newcommand{\diff}[1]{\ensuremath{\operatorname{D}\!{#1}}} % Jacobian
\newcommand{\inner}[2]{\left\langle #1, #2 \right\rangle} % inner product
\newcommand{\norm}[1]{\left\lVert#1\right\rVert} % norm
\newcommand{\cpl}[1]{\overline{#1}} % complement
\renewcommand{\v}[1]{\mathbf{#1}} % vector
\newenvironment{amatrix}[1]{% augumented matrix - make sure to have # columns less than required amount
  \left(\begin{array}{@{}*{#1}{c}|c@{}}
}{%
  \end{array}\right)
}
%-----------------------------------------------------------------------------------------------------------------
% Define theorem environments, along with a custom proof environment
\usepackage[thref, thmmarks,amsmath]{ntheorem}
\newcommand{\itref}[1]{\textit{\thref{#1}}}

\newtheorem{theorem}{Thm.}[section]
\newtheorem{lemma}[theorem]{Lemma}
\newtheorem{definition}[theorem]{Def'n.}
\newtheorem{corollary}[theorem]{Cor.}
\newtheorem{proposition}[theorem]{Prop.}

\theorembodyfont{\upshape}
\newtheorem{remark}[theorem]{Rmk.}
\newtheorem{exercise}[theorem]{Exc.}
\newtheorem{example}[theorem]{Ex.}
\theoremseparator{}
\theoremindent0.0cm
\theoremstyle{nonumberplain}
\theoremheaderfont{\scshape}
\theoremsymbol{$\square$}
\newtheorem{proof}{Proof}

%-----------------------------------------------------------------------------------------------------------------
% Define Document Variables
\newcommand{\assignmentname}{Course Notes}
\newcommand{\classname}{Introduction to Abstract Algebra}
\newcommand{\semester}{BSM Fall 2018}

% Define a title page for the document
%----------------------------------------------------------------------------------------------------------------------
% Define headings for each page
\usepackage{fancyheadings}
\pagestyle{fancy}
\lhead{Alex Rutar\\arutar@uwaterloo.ca}
\rhead{\classname: \assignmentname\\\semester}
\cfoot{\thepage}
\setlength{\headheight}{50pt}
%----------------------------------------------------------------------------------------------------------------------
\begin{document}
\pagenumbering{roman}
\begin{titlepage}
    \centering
    \vspace{5cm}
    {\huge\textbf{\assignmentname}\par} % Assignment Name
    \vspace{2cm}
    {\Large\textbf{\classname}\par} % Class
    \vspace{3cm}
    {\Large\textit{Alex Rutar}\par}

    \vfill

% Bottom of the page
    {\large \semester \par} % Due Date
\end{titlepage}
%----------------------------------------------------------------------------------------------------------------------
% \newpage\null\thispagestyle{empty}\textit{This page is left intentionally blank.}\newpage
\pagenumbering{roman}
\tableofcontents
\pagenumbering{arabic}
\chapter{Fundamentals of Groups}
\section{Principles}
In general, algebraic structures require three properties:
\begin{itemize}
    \item A set
    \item Operations on the set
    \item Properties of these operations
\end{itemize}
We develop theories and want to look at examples to demonstrate these properties.
This course will focus on propeties of rings and groups.

\subsection{Rings}
A ring consists of a set along with two binary operations which satisfy $(R,+,\cdot)$.
Then for all $a,b,c\in R$,
\begin{enumerate}[nolistsep]
    \item $(a+b)+c=a+(b+c)$
    \item $a+b=b+a$
    \item $\exists 0\in R$ so that $a+0=a$
    \item $\forall a\in R$, there exists $b\in R$ so that $a+b=0$
    \item $(a\cdot b)\cdot c=a\cdot(b\cdot c)$
    \item $a\cdot(b+c)=a\cdot b+a\cdot c$ and $(a+b)\cdot c=a\cdot c+b\cdot c$
\end{enumerate}
There are some common examples:
\begin{enumerate}
    \item Rings of numbers
        \begin{enumerate}
            \item $\Z,\Q,\R,\C$
            \item $\Z[\sqrt{2}]=\{a+b\sqrt{2}\mid a,b\in\Z\}$ 
            \item $\Q[\sqrt[3]{2}]=\{a+b\sqrt[3]{2}+c\sqrt[3]{4}|a,b,c\in\Q\}$
        \end{enumerate}
    \item Rings of polynomials
        \[\Z[x]=\{a_nx^n+a_{n-1}x^{n-1}+\cdots+a_1x+a_0|\forall a_i\in \Z\}\]
        $\Q[x],\R[x],\C[x],\Z[x,y]$ etc.
    \item Rings of functions, such that $C[a,b]$
    \item Rings of matrices $M_n(\Z)$: all $n\times n$ square matrices with integer entries (more generally matrices with any entries in a ring).
    \item Given any set $X$, consider $\mathcal{P}(X)$ and define the symmetric difference
        \[A\oplus B=(A\cup B)\setminus(A\cap B)\]
        Then $(\mathcal{P}(X),\oplus,\cap$ is a ring.
        Interestingly, $A=-A$ in this ring.
\end{enumerate}
A ring with identity means we have some $1\neq 0$ so that $a\cdot 1=1\cdot a=a$.
A division ring is a ring with identity such that all nonzero elements have a multiplicative inverse.
A field is a commutative division ring $\Q,\R,\C,\Q[\sqrt{2}]$.
\subsection{Groups}
\begin{definition}
    A group is a set $G$ together with an operation $*$ which satisfies
    \begin{enumerate}
        \item $(a*b)*c=a*(b*c)$
        \item $\exists e\in G:a*e=a=e*a$
        \item $\forall a\in G\exists b\in G:a*b=e=b*a$
    \end{enumerate}
\end{definition}
Here are some common examples of groups
\begin{enumerate}
    \item Additive groups:
        \begin{enumerate}
            \item If $(R,+,\cdot)$ is a ring, then $(R,+)$ is a (commutative) group.
            \item If $V$ is a vector space, then $(V,+)$ is a group
        \end{enumerate}
    \item Multiplicative groups:
        \begin{enumerate}
            \item $R$ is a ring with identity, and write
                \[R^\times=\{a\in R\mid \exists b\text{ s.t. }a\cdot b = 1 = b\cdot a\}\]
                in other words the elements having a multiplicative inverse.
                These are called the \textbf{units} of the ring, and $R^\times$ is called the \textbf{unit group} or the \textbf{multiplicative group} of $R$.
            \item $\Z^\times=\{1,-1\}$, $\Q^\times=\Q\setminus\{0\}$ (similarly for $\R,\C$).
            \item $M_n(\R)^\times = \GL_n(\R)=\{A\in M_n(\R)\mid \det A\neq 0\}$.
            \item $M_n(\Z)^\times = \GL_n(\Z)=\{A\in M_n(\Z)|\det A=\pm 1\}$.
        \end{enumerate}
    \item Matrix groups: matrices under addition and multiplication
    \item Composition of permutations.
        Let $T$ be any set, and $A:T\to T$ be bijective.
        Let $S_T$ be the collection of all permutations on $T$.
        Then $(S_T,\circ)$ (composition action) forms a group.

        We write $S_n=S_{\{1,2,\ldots,n\}}$, the group of permutations on $n$ elements.
        We can notate the elements of $S_n$ by writing
        \[
            \begin{pmatrix}
                1&2&\cdots&n\\
                f(1)&f(2)&\cdots&f(n)
            \end{pmatrix}
        \]
        Clearly $|S_n|=n!$.
\end{enumerate}
\subsection{The group $\Z_m$}
\begin{definition}
    Let $\sim$ be an equivalence relation.
    We then define the \textbf{quotient group} $G/\sim$ given by the equivalence classes of elements in $G$.
\end{definition}
To construct $\Z_m$, we define $\Z_m=\Z/\sim$ where $a\sim b$ if $a\cong b\pmod{m}$.
Since we have a division algorithm in $\Z$, for any $d\in\Z$, we can write $d=tm+r$ with $0\leq r\leq m-1$.
Thus $\overline{d}=\overline{r}$, so we can represent $\Z_m=\{\overline{0},\overline{1},\ldots,\overline{m-1}\}$.
As a result we usually do not bother writing $\overline{\cdot}$.
\begin{proposition}
    We have $\overline{a}+\overline{b}=\overline{a+b}$ and $\overline{a}\cdot\overline{b}=\overline{ab}$.
\end{proposition}
\begin{proof}
    Obvious.
\end{proof}
\begin{theorem}
    $\Z_m^\times=\{\overline{a}\mid\gcd(a,m)=1\}$.
\end{theorem}
\begin{proof}
    Assume $\overline{a}\in\Z^\times_m$ so there exists $\overline{x}$ with $\overline{x}\cdot\overline{a}=1$.
    Then $\overline{xa}=\overline{1}$ so $xa\cong 1\pmod{m}$ so $m|xa-1$.
    Let $d=\gcd(a,m)$ so $d|a$ and $d|m$.
    Thus $d|xa-1$ and $d|xa$ so $d|1$ and $\gcd(a,m)=1$.

    Conversely, suppose $\gcd(a,m)=1$.
    Then by B\'ezout's Lemma, get $x,y$ so that $xa+ym=1$, so $xa\cong 1\pmod{1}$ and $\overline{xa}=\overline{1}$ and $\overline{x}\overline{a}=\overline{1}$ and we have our multiplicative inverse.
\end{proof}
We thus have $|\Z_m^\times|=\phi(m)$.
\section{Basics of Groups}
\subsection{Functions between Groups}
\begin{definition}
    Let $(G,\Diamondblack)$, $(H,\star)$ be groups.
    A mapping $f:G\to H$ is called an \textbf{homomorphism} if
    \[f(u\Diamondblack v)=f(u)\star f(v)\]
    If $f$ is also a a bijection, then we call $f$ an \textbf{isomorphism}.
\end{definition}
\begin{proposition}
    $G$ and $H$ are isomorphic if and only if their Cayley Tables are the same up to permutation of elements.
\end{proposition}
\begin{proof}
    Obvious.
\end{proof}
\section{Examples of Finite Groups}
\subsection{Group Definitions}
\begin{definition}
    We say that $(G,*)$ with $*:G\times G\to G$ is a \textbf{group} if for all $a,b,c\in G$
    \begin{enumerate}
        \item $(a*b)*c=a*(b*c)$
        \item $\exists e\in G:\quad a*e=a=e*a$
        \item $\exists u\in G:\quad a*u=e=u*a$
    \end{enumerate}
\end{definition}
We have our first basic proposition:
\begin{proposition}
    The identity and inverses are unique.
\end{proposition}
\begin{proof}
    If $e,f$ are both identities, then $e=e*f=f$.
    If $u,v$ are both inverses of $x$, then $u*(x*v)=u*e=u$ and $(u*x)*v=e*v=v$ so $u=v$.
\end{proof}
\begin{definition}
    If $ab=ba$ for all $a,b\in G$ then we say that $G$ is \textbf{commutative}.
\end{definition}
\begin{definition}
    Let $G$ be a group with $G=\{g_1,g_2,\ldots,g_n\}$.
    Then the \textbf{Cayley Table} for $G$ is the matrix $M\in M_n(G)$ where $M_{ij}=g_ig_j$.
\end{definition}
\begin{proposition}
    In each column or row, each element occurs exactly once.
    Furthermore, if $M_{ij}=e$, then $M_{ji}=e$.
\end{proposition}
\begin{proof}
    This follows directly by left or right cancellation, and by commutativity of the elements with their inverse.
\end{proof}
\subsection{Cyclic Groups}
\begin{definition}
    The \textbf{order of an element} $g\in G$ is $o(g):=\left\lvert\{g^d|d\in\Z\}\right\rvert$.
    The \textbf{order of a group} $G$ is $|G|$.
\end{definition}
We certainly have $o(g)\leq|G|$ for any $g\in G$.
Equality holds when $o(g)=\infty$ and $G$ is countable, or $G=\{g^d:d\in\Z\}$.
\begin{definition}
    A collection $H=\{g_1,g_2,\ldots,g_k\}$ \textbf{generates} $G$ if we can write any $g\in G$ as a product of elements in $H$.
\end{definition}
\begin{definition}
    We say that $G$ is \textbf{cyclic} if $G=\{g^d:d\in\Z\}$ for some $g\in G$.
    Equivalently, it is generated by a set of cardinality one.
\end{definition}
\begin{example}
    Note that $\Z_{13}^\times$ is cyclic with generator $2$.
\end{example}
\begin{lemma}
    If $o(g)$ is finite and $d\in\Z$, then
    \[o(g^d)=\frac{o(g)}{\gcd(o(g),d)}\]
\end{lemma}
\begin{proof}
    Let $o(g)=K$ and $t=\gcd(K,d)$ and write $K=tK_1$ and $d=td_1$ with $K_1,d_1$ coprime.
    Thus $o(g^d)$ is the smallest positive integer $l$ with $(g^d)^l=1$.
    But then $(g^d)^l=1\Leftrightarrow g^{dl}=1\Leftrightarrow o(g)|dl$ and $k|dl$, that is $tK_1|td_1l$ and $k_1|d_1l$.
    Thus $K_1|l$ so the smallest positive intger $l$ is $K_1$ and $o(g^d)=K_1=\frac{o(g)}{\gcd(o(g),d)}$ as desired.
\end{proof}
\subsection{Permutation Groups}
Recall that $S_n$ is the symmetric group of degree $n$, consisting of all permutations of $[n]$.
Thus $|S_n|=n!$.
Instead of using the matrix form, we can write the permutation group using the cycle form.
\begin{example}
    Write
    \[f=\begin{pmatrix}1&2&3&4&5&6&7&8&9\\4&7&3&1&2&9&8&5&6\end{pmatrix}=(14)(2785)(3)(69)\]
    We can also write $(14)(2785)(69)$, in other words excluding elements which map to themselves.
\end{example}
In general, a cycle $(a_1a_2\ldots a_k)$ indicates that $a_1f=a_2$, $a_2f=a_3$,\dots,$a_kf=a_1$.
In $S_n$, each permutation can be expressed in a cycle form (using disjoint cycles).
The cycle form is unique up to ordering within the cycles, and ordering among the cycles.
\begin{example}
    In $S_5$, the possible cycle structures are
    \[I,(ab),(abc),(abcd),(abcde),(ab)(cd),(ab)(cde)\]
    We then have
    \begin{align*}
        o(I) &= 1\\
        o((ab)) &= 2\\
        o((abc)) &= 3\\
        o((abcd)) &= 4\\
        o((abcde)) &= 5\\
        o((ab)(cd)) &= 2\\
        o((ab)(cde)) &= 6
    \end{align*}
    For $f=(abc)$, $f^2=(abc)(abc)=(acb)$, $f^3=(abc)(acb)=abc$.
    For $f=(abcd)$, $f^2=(ac)(bd)$, $f^3=(abdc)(ac)(bd)(adcb)$, and $f^4=(abcd)(adcb)=(abcd)$.

    If $f=(a_1a_2\ldots a_k)$, $o(f)=k$.
\end{example}
\begin{proposition}
    Suppose $f=\gamma_1\gamma_2\ldots\gamma_i$ for disjoint cycles.
    Then $o(f)=lcm(o(\gamma_1),o(\gamma_2),\ldots,o(\gamma_i))$.
\end{proposition}
\begin{proof}
    Note that the $\gamma_i$ commute, so that
    \begin{align*}
        f^d=I &\Leftrightarrow (\gamma_1\gamma_2\ldots\gamma_i)^d=I\\
              &\Leftrightarrow \gamma_1^d\gamma_2^d\ldots\gamma_i^d=I\\
              &\Leftrightarrow \gamma_i^d=I\quad\forall i
    \end{align*}
    The last line holds since the $\gamma_i^d$ operates on disjoint sets.
    Thus we have our formula, as desired.
\end{proof}
\subsection{Dihedral Groups}
Fix a regular polygon with $n$ vertices.
Let $D_n$ be the collection of rigid motions with map the regular $n-$polygon to itself.
Since $r^n=1$ and $s^2=1$, we have
\[D_n=\{1,r,r^2,\ldots,r^{n-1},s,sr,sr^2,\ldots,sr^{n-1}\}\]
Thus $|D_n|=2n$.
We can compute the oprations on $D_n$:
\begin{align*}
    r^a\cdot r^b &= r^{a+b}\\
    sr^a\cdot r^b &= sr^{a+b}\\
    r^a\cdot sr^b &= sr^{b-a}\\
    sr^a\cdot sr^b &= r^{b-a}
\end{align*}
Thus $o(sr^a)=2$ and $o(r^a)$ is given by the usual formula.
\section{Subgroups}
\begin{definition}
    A subset $H$ of a group $G$ is called a \textbf{subgroup} if $H$ is also a group with the same operation.
    We write $H\leq G$.
\end{definition}
For example, $(\Z,+)\leq(\Q,+)\leq(\R,+)\leq(\C,+)$.
\end{document}
