\documentclass[12pt, a4paper]{article}
\usepackage[ascii]{inputenc}
\usepackage[left=2cm,right=2cm,top=2cm,bottom=4cm]{geometry}
\usepackage[protrusion=true,expansion=true]{microtype}

\usepackage{amsmath}
\usepackage{amsfonts}
\usepackage{amssymb}
\usepackage{kpfonts}
\usepackage{tikz, pgfplots}
\pgfplotsset{compat=1.13}

\DeclareMathOperator{\N}{\mathbb{N}}
\DeclareMathOperator{\Q}{\mathbb{Q}}
\DeclareMathOperator{\Z}{\mathbb{Z}}
\DeclareMathOperator{\R}{\mathbb{R}}
\DeclareMathOperator{\C}{\mathbb{C}}
\DeclareMathOperator{\F}{\mathbb{F}}
\DeclareMathOperator{\im}{im}
\DeclareMathOperator{\re}{re}
\DeclareMathOperator{\proj}{proj}

\usepackage{graphicx}
\usepackage{enumitem}
\setenumerate[1]{label=\textbf{\arabic*.}}

%-----------------------------------------------------------------------------------------------------------------
% Some fancy macros // May eventually move these into separate files or something and merge when building template
\renewcommand{\d}[1]{\ensuremath{\operatorname{d}\!{#1}}} % dx macro for integrals
\newcommand{\hess}[1]{\ensuremath{\operatorname{H}\!{#1}}} % Hessian
\newcommand{\diff}[1]{\ensuremath{\operatorname{D}\!{#1}}} % Jacobian
\newcommand{\inner}[2]{\left\langle #1, #2 \right\rangle} % inner product
\newcommand{\norm}[1]{\left\lVert#1\right\rVert} % norm
\newcommand{\cpl}[1]{\overline{#1}} % complement
\renewcommand{\v}[1]{\mathbf{#1}} % vector
\newenvironment{amatrix}[1]{% augumented matrix - make sure to have # columns less than required amount
  \left(\begin{array}{@{}*{#1}{c}|c@{}}
}{%
  \end{array}\right)
}
%-----------------------------------------------------------------------------------------------------------------
% Define theorem environments, along with a custom proof environment
\usepackage[thref, thmmarks,amsmath]{ntheorem}
\newcommand{\itref}[1]{\textit{\thref{#1}}}

\theoremseparator{}
\theoremindent0.0cm
\theoremstyle{nonumberplain}
\theoremheaderfont{\scshape}
\theorembodyfont{\upshape}
\theoremsymbol{$//$}
\newtheorem{solution}{Sol'n}
\theoremsymbol{$\blacksquare$}
\newtheorem{proof}{Proof}

%-----------------------------------------------------------------------------------------------------------------
% Define Document Variables
\newcommand{\assignmentname}{Condensed Notes}
\newcommand{\classname}{Algebra 1}
\newcommand{\duedate}{Monday, December 17}

%-----------------------------------------------------------------------------------------------------------------
% Define headings for each page
\usepackage{fancyheadings}
\pagestyle{fancy}
\lhead{Alex Rutar\\ID 2065 1307}
\chead{\classname}
\rhead{\assignmentname\\\duedate}
\cfoot{\thepage}
\setlength{\headheight}{50pt}
%-----------------------------------------------------------------------------------------------------------------
\begin{document}
\section{Required Proofs}
\begin{enumerate}
    \item \textbf{For any subgroup $H\leq G$, the following hold:}
        \begin{enumerate}[nolistsep]
            \item $|Hg|=|H|$
            \item $Hg=H\Leftrightarrow g\in H$
            \item \textbf{Any two right cosets of $H$ are equal or disjoint.}
            \item $Hx=Hy\Leftrightarrow xy^{-1}\in H$
        \end{enumerate}
        \begin{proof}
            Recall that $Hg=\{hg:h\in H\}$.
            We thus have
            \begin{enumerate}[nolistsep]
                \item Let's see that the map $\phi:H\to Hg$ given by $h\mapsto hg$ is a bijection.
                    It is injective: if $h_1g=h_2g$, then multiplying on the right by $g^{-1}$ implies that $h_1=h_2$.
                    It is surjective: if $x\in Hg$, then $x=h_1g$ for some $h_1\in H$.
                    But then $x=\phi(h_1)$.
                \item If $Hg=H$, clearly $g\in Hg$ so $g\in H$.
                    Conversely, if $g\in H$, then since $H$ is closed under multiplication (it is a subgroup), $Hg=H$.
                \item If $Hg_1$ and $Hg_2$ are not disjoint, let $x\in Hg_1$ and $Hg_2$.
                    Then $x=h_1g_1=h_2g_2$ so $h_1^{-1}h_2g_2=g_1$.
                    Now for any $hg_1\in Hg_1$, we have $hg_1=hh_1^{-1}h_2g_2\in Hg_2$ so $Hg_1\subseteq Hg_2$.
                    Since $|Hg_1|=|Hg_2|$ by (1), equality must hold.
                \item First suppose $Hx=Hy$.
                    Then to each $h\in H$, there exists $h'$ so $hx=h'y$; that is, $xy^{-1}=h^{-1}h'\in H$.
                    Conversely, if $xy^{-1}\in H$, then $x=xy^{-1}y\in Hy$ so $x\in Hx$ and $x\in Hy$ and by (3), $Hx=Hy$.
            \end{enumerate}
        \end{proof}
    \item \textbf{The conjugacy relation is an equivalence relation on $G$, and for any $g\in G$, $|C_g|\cdot|G_G(g)|=|G|$.}
        \begin{proof}
            Recall that $x\sim y$ if and only if there exists $g\in G$ so $g^{-1}xg=y$.
            \begin{enumerate}[nolistsep]
                \item \textit{Reflexive:} $x\sim x$ since $1^{-1}x1=x$.
                \item \textit{Symmetric:} If $x\sim y$, then $g^{-1}xg=y$ and $(g^{-1})^{-1}yg^{-1}=x$ so $y\sim x$.
                \item \textit{Transitive:} If $x\sim y$ and $y\sim z$, then $g^{-1}xg=y$ and $h^{-1}y=z$, so $(gh)^{-1}xgh=z$ and $x\sim z$.
            \end{enumerate}
            Recall that $C_G(g)\leq G$.
            It suffices to show $[G:C_G(g)]=|C_g|$: in particular, I claim that the map from right cosets of $C_G(g)$ to conjugate elements of $g$ given by $C_G(g)h\mapsto h^{-1}gh$ is a bijection.
            Let's first see that it is well-defined and injective.
            We have
            \begin{align*}
                C_G(g)h_1=C_G(g)h_2&\Longleftrightarrow h_1h_2^{-1}\in C_G(g)\\
                                   &\Longleftrightarrow h_1h_2^{-1}g=gh_1h_2^{-1}\\
                                   &\Longleftrightarrow h_2^{-1}gh_2=h_1^{-1}gh_1
            \end{align*}
            It is also surjective: if $hg^{-1}h$ is an arbitrary conjugate element, then it is the image of $C_G(g)h$.
            Thus the map is bijective, so
            \begin{equation*}
                [G:C_G(g)]=|C_g|\Longrightarrow \frac{|G|}{|C_G(g)|}=|C_g|
            \end{equation*}
            and the desired result holds.
        \end{proof}
    \item \textbf{Subgroups of cyclic groups are also cyclic.}
        \begin{proof}
            Let $G=\langle g\rangle$ be cyclic, and let $H\leq G$.
            If $H=\{1\}$ it is certainly cyclic; otherwise, let $n\neq 0$ be minimal so that $g^n\in H$.
            I claim that $H=\langle g^n\rangle$.
            Certainly $\langle g^n\rangle\subseteq H$ by closure under multiplication.
            If $h\in H$ is arbitrary, write $h=g^{kn+r}$ for some $k,r\in\N$ with $r<n$.
            But then $g^r=h(g^{k})^{-n}\in H$, so by minimality of $n$, we must have $r=0$.
            Thus $h=(g^n)^k\in\langle g^n\rangle$ so $H\subseteq\langle g^n\rangle$ and equality holds, as desired.
        \end{proof}
    \item \textbf{Groups of order $p^2$ (with $p$ any prime) are commutative.}
        \begin{proof}
            First recall that $G$ is a disjoint union of its conjugacy classes.
            Let's first see that $Z(G)=\{g\in G:|C_g|=1\}$.
            If $|C_g|=1$, then $C_g=\{g\}$ so $x^{-1}gx=g$ and $gx=xg$ for any $x\in G$.
            Similarly, if $g\in Z(G)$, then $gx=xg$ for any $x\in G$ so $x^{-1}gx=g$ and $C_g=\{g\}$.
            Thus $G$ is a disjoint union of its center along with its non-trivial conjugacy classes (this is commonly referred to as the \textit{class equation}).
            Recall as well that $|C_g|$ divides $|G|$ for all $g\in G$.

            Let $|G|=p^2$ and write $|G|=|Z(G)|+\sum_{i=1}^k |C_{g_i}|$ where the $C_{g_i}$ are disjoint non-trivial conjugacy classes.
            Since $|C_{g_i}|>1$, we must have $|C_{g_i}|\equiv 0\pmod{p}$.
            Thus $|Z(G)|\equiv 0\pmod{p}$, and since $|Z(G)|\geq 1$, we have $|Z(G)|=p$ or $|Z(G)|=p^2$.

            If $|Z(G)|=p^2$, it is clear that $G$ is commutative, so suppose $|Z(G)|=p$.
            Let $x\in G\setminus Z(G)$, so $Z(G)\lneq C_G(x)$.
            Thus $p$ divides $|C_G(x)|$ and $|C_G(x)|\geq p+1$, so $|C_G(x)|=p^2$.
            Thus $C_G(x)=G$ and $x\in Z(G)$, a contradiction.
        \end{proof}
    \item \textbf{First Isomorphism Theorem: for any homomorphism $\phi:G\to H$ of groups, $G/\ker(\phi)\cong\im(\phi)$.}
        \begin{proof}
            Consider the map $\alpha$ from right cosets of $\ker(\phi)$ to $\im(\phi)$ given by $\ker(\phi)h=\phi(h)$.
            First, let's check that $\alpha$ is well-defined and injective.
            By properties of homomorphisms,
            \begin{align*}
                \ker(\phi)h_1=\ker(\phi)h_2 &\Longleftrightarrow h_1h_2^{-1}\in\ker(\phi)\\
                                            &\Longleftrightarrow \phi(h_1h_2^{-1})=1\\
                                            &\Longleftrightarrow \phi(h_1)\phi(h_2)^{-1}=1\\
                                            &\Longleftrightarrow \phi(h_1)=\phi(h_2)
            \end{align*}
            and to see surjectivity, if $y\in\im(\phi)$, then $y=\phi(h)$ and $y=\alpha(\ker(\phi)h)$.

            It remains to check that $\alpha$ is a homomorphism.
            Indeed,
            \begin{align*}
                \alpha(\ker(\phi)h_1\ker(\phi)h_2) &= \alpha(\ker(\phi)(h_1h_2)\\
                                                   &= \phi(h_1h_2)\\
                                                   &= \phi(h_1)\phi(h_2)\\
                                                   &= \alpha(\ker(\phi)h_1)\alpha(\ker(\phi)h_2)
            \end{align*}
            as required.
        \end{proof}
    \item \textbf{If $M,N$ are normal subgroups in a group $G$ with $M\cap N=\{1\}$, then $mn=nm$ for all $m\in M$ and $N\in N$.
            If we assume additionally that $MN=G$, then $G\cong M\times N$.
        }
        \begin{proof}
            % Recall that $M$ is normal if $Mx=xM$ for all $x\in G$.
            To show that $mn=nm$, it suffices to show that $m^{-1}n^{-1}mn\in M\cap N=\{1\}$.
            Since $M$ is normal and $m\in M$, $n^{-1}mn\in M$ so $m^{-1}n^{-1}mn\in M$.
            Similarly, $m^{-1}n^{-1}m\in N$ since $N$ is normal, so $m^{-1}n^{-1}mn\in N$ as well.

            Now, let's define $\phi:M\times N\to G$ by $\phi(m,n)=m\cdot n$.
            Since $M\cdot N=G$, $\phi$ is surjective, so let's check injectivity.
            We have using the identity proved earlier
            \begin{align*}
                \phi(m_1,n_1)=\phi(m_2,n_2) &\Longrightarrow m_1n_1=m_2n_2\\
                                            &\Longrightarrow m_2^{-1}m_1=n_2n_1^{-1}\\
                                            &\Longrightarrow m_1m_2^{-1},n_1n_2^{-1}\in M\cap N\\
                                            &\Longrightarrow m_1m_2^{-1}=1,n_1n_2^{-1}\\
                                            &\Longrightarrow (m_1,n_1)=(m_2,n_2)
            \end{align*}
            so it remains to show that $\phi$ is a homomorphism.
            Indeed,
            \begin{align*}
                \phi((m_1,n_1)\cdot(m_2,n_2)) &= \phi(m_1m_2,n_1n_2)\\
                                              &= m_1m_2n_1n_2\\
                                              &= m_1n_1m_2n_2\\
                                              &= \phi(m_1,n_1)\phi(m_2,n_2)
            \end{align*}
            by the claim proven earlier, as required.
        \end{proof}
    \item \textbf{A commutative simple ring is either a field or a zero-ring.}
        \begin{proof}
            If $R=\{0\}$ then it is certainly a zero-ring, so suppose $R\neq\{0\}$.
            First suppose $R$ has zero divisors and get $a,b\neq 0$ with $a\cdot b=0$.
            Define $N(a)=\{x\in\R:a\cdot x=0\}$.
            Note that $N(a)\triangleleftleq R$: if $x,y\in N(a)$ then $(x+y)a=xa+ya=0$, and for any $r\in R$, $(rx)a=r(xa)=0$.
            Since $b\neq 0$, $b\in N(a)$, so $N(a)=R$ since $R$ is simple.
            Now define $N=\{x\in R:xR=0\}$.
            Again, $N\triangleleftleq R$ since $(x+y)R=xR+yR=0$ and $(ax)R=a(xR)=0$.
            Note that $a\in N$ and $a\neq 0$, so as before, $N=R$ and $R$ is a zero-ring.

            Otherwise, we assume $R$ has no zero divisors.
            Let $a\neq 0$, so $\{0\}\neq Ra\triangleleq R$ and $Ra=R$.
            Since $a\in R$, get $e\in R$ so that $ea=a$.
            Then if $b$ is arbitrary, $ba=bea$ so $(b-be)a=0$ and since $a\neq 0$, $b=be$.
            Since $R$ is commutative, $be=eb=b$ so $e\in R$ is an identity element
            Now if $x\neq 0$ is arbitrary, $Rx=R$ so there exists $y\in R$ so $yx=e$, so every $x$ has an inverse.
            Thus $R$ is a field.
        \end{proof}
    \item \textbf{In an integral domain, every prime element is irreducible.
            In a prinicipal ideal domain, $\gcd(a,b)$ always exists and can be expresed as $xa+yb$ with some $x,y\in R$.
            In a principal ideal domain, every irreducible element is prime.
        }
        \begin{proof}
            Let $p\in R$ be prime and suppose $d|p$.
            Get $x$ so that $dx=p$; then, since $p$ is prime, $p|x$ or $p|d$.
            If $p|d$, then $p\sim d$; if $p|x$, get $x$ so that $x=py$.
            Then $dpy=p$ so $(dy-1)p=0$ and since $R$ is integral, $dy=1$ so $d$ is a unit.
            
            Fix elements $a,b\in R$ and consider the ideal $I=\{xa+yb:x,y\in R\}$.
            This is an ideal: $x_1a+y_1b+x_2a+y_2b=(x_1+x_2)a+(y_1+y_2)b\in I$ and $r(xa+yb)=(rx)a+(ry)b$.
            Since $R$ is a PID, $I=(d)$; note that $d|a$ and $d|b$.
            Since $d\in I$, $d=xa+yb$ for some $x,y\in R$; thus, if $c|a$ and $c|b$, then $c|xa+yb=d$, so $d$ is a greatest common divisor.
            If $d'$ is any other greatest common divisor, then $d'=ud$ so $d'=(ux)a+(uy)b$.

            Finally, suppose $q\in R$ is irreducible and $q|ab$.
            Note that $\gcd(q,a)|q$ so either $q\sim\gcd(q,a)$ or $1\sim\gcd(q,a)$.
            In the first case, $q|a$.
            In the second case, there exists $x,y$ so that $1=xq+ya$.
            Then $b=xqb+yab$ and $q|xqb$ and $q|yab$, so $q|b$.
        \end{proof}
    \item \textbf{Every Euclidean domain is a principal ideal domain.}
        \begin{proof}
            Let $J$ be an arbitrary ideal and let $d\in J$ be such that $N(d)$ is minimal.
            Clearly $(d)\subseteq J$; it suffices to show that $J\subseteq (d)$.
            If $x\in J$ is arbitrary, write $x=qd+r$ with $N(r)<N(d)$.
            Noce that $r=x-qd\in J$, so by minimality of $d$, $r=0$.
            Thus $x=qd\in (d)$.
        \end{proof}
\end{enumerate}
\section{All Definitions}
\end{document}
